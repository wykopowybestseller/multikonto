\chapter{Życie XVI}
\chapterAuthor{BQP}

„Nowy dzień, nowa taśma, pozytywne crew. Może faktycznie nic nie stało się tu” -- moje pierwsze słowa, jakie usłyszałem w~nowym ciele. Siedziałem przed komputerem, słuchając jakieś internetowego radia. Przede mną strona w~necie otwarta na jakimś durnym portalu, chyba satanistycznym, bo przeglądając ją trafiałem co chwilę na przerobione zdjęcia jakiegoś papieża. Byłem nawet zalogowany jako „BQP”. Pewnie znowu trafiłem w~jakieś zjebane życie, które spierdolę jeszcze bardziej. Nieważne, ważne jest co innego. Drogi czytelniku, jeżeli to czytasz należy ci się kilka słów wyjaśnień. Trzymasz w~swojej ręce mój dziennik. Nie wiem jak go znalazłeś, ale po przeczytaniu odłóż go z~powrotem. Postanowiłem pisać te dzienniki, by się nie pogubić. Kiedy wpadam w~nowe ciało, moje stare wspomnienia blakną, zamieniane przez wspomnienia mojego nosiciela. W~mojej głowie jest chaos, co widać po wcześniejszych wpisach. Postaram się więc posegregować to, co stało się w~moich poprzednich wcieleniach, gdyż miejsce w~tym dzienniku zaczyna się kończyć. A~potem schowam go wraz z~innymi, by znaleźć go w~następnym wcieleniu. I~dopisać kolejne historie.

Moja „przygoda” zaczęła się w~roku 2013 w~wieku lat 18. Śmiercią. W~dzień moich urodzin. Przerwa między wchodzeniem w~inne ciała była różna. Czasem trwała kilka minut, dni a~czasem lat. Gdy umierałem trafiałem do kolejnego ciała. Po jakimś czasie moja „moc” zaczęła nabierać siły, bo zacząłem cofać się w~czasie. Bitwa pod Waterloo, czasy starożytnej Japonii itp. Zawsze dla mnie było zagadką w~jakim czasie powrócę do żywych. Choć czy ja tak naprawdę kiedykolwiek umarłem?

Mam czas, jest godzina 8:30, a~według wspomnień mojego nosiciela do pracy jechałem na 14.

Od początku więc, gdyż w~mojej głowie mam bałagan, wielu wspomnień z~poprzednich wcieleń nie pamiętam dokładnie, a~wydają się kluczowe. Co jednak wiem na 100\%? Wiem, że żyją ludzie tacy jak ja. I~tacy, którzy na nas polują. Nie wiem dlaczego. Nigdy się nie dowiedziałem. Jasna cholera, zawsze więcej pytań niż odpowiedzi! Dlaczego ja? Jakie jest źródło tajemniczej mocy, która wciąż miota mną od jednego żywota do drugiego? Po co to robi? Czy ciała są przypadkowe, czy może jest w~tym jakiś ukryty sens, który mi ucieka? Jak się tego dowiedzieć, nie wiem. Raz nawet byłem tajnym agentem, który badał sprawy nadprzyrodzone. Chyba to badał… Muszę sobie przypomnieć. A~w~międzyczasie przygotuję się do roboty.

Spojrzałem w~lustro. O~kurwa. Przestałem patrzeć w~lustro. To chociaż chatę pozwiedzam. Też chujnia. Zwykła standardowa chata wolnostojąca. Kilka pokojów, mini salon, kuchnia, dwa kible. Jeden na piętrze obok sypialni i~mniejszy kibel na parterze, który jest moim kiblem ulubionym. Może nie moim, ale „poprzedniego” właściciela, którego ciało teraz bezczelnie, choć nie celowo okupowałem. Spojrzałem przez okno. Podwórze było małe i~nie było na nim nic ciekawe… Albo jednak coś było. Alfa Romeo 159. Ładny samochód. Zresztą co ja mówię, to najpiękniejszy skurwysyn jakiego miałem w~jakimkolwiek swoim życiu. Ubrałem się w~jakieś niemodne ciuchy i~pojechałem do roboty. Potrzebowałem wspomnień, co robię? Gdzie? I~temu podobne pierdoły, żeby gładko wejść w~swoje nowe życie. Pamiętam swój pierwszy dzień w~pracy. A~raczej dzień przed nim. Dzień, kiedy jechałem na badania lekarskie….

Jadę sobie na badania, bo zaczynam nową pracę i~będę nakurwiał wózkami wysokościowymi (te co mogą podnosić towar na kilkanaście metrów do góry). Miałem jechać więc do psychologa na psychotesty (przecież tyle się słyszy o~tych kierowcach wózków, którzy specjalnie nabijają kolegów na widły). Prewencyjnie poszedłem strzelić kupę, co by mnie nie zaskoczyła, niczym rowerzysta bez odblasków za zakrętem. Droga daleka, bo całe 40 km ale gotów jestem, więc jadę. Dojeżdżając do celu i~kurwa, jak mnie sranie nie weźmie, to zobaczyłem gwiazdki przed oczyma z~wysiłku spinając poślady. Chuj, nie ma odwrotu. Żebym jeszcze przyjechał prędzej, to może bym coś ogarnął w~tym szpitalu, ale nie. Wchodzę, punktualnie, co do minuty i~widzę, że drzwi otwarte i~już na mnie babka czeka (dobre 8/10 ale miała ok trzydziestki, więc odpada), więc wchodzę. Standardowo gadka szmatka i~po chwili pierwsze zadanie. Musze położyć nogi na dwóch pedałach. Jak się zaświeci zielone światło to mam wcisnąć prawą nogą, a~jak czerwone to lewą. Pani doktor mówi, że czas start i~mam nakurwiać tak przez 3 minuty. No to jazda! Zadanie okazało się jednak wymagające, bo pedały były wyjebane jak sprzęgła w~samochodach poznańskiego WORDu, a~liczył się czas reakcji. Wyobraziłem sobie, że pod jedną nogą mam ryj Donalda, a~pod drugą Vincenta i~to okazało się skuteczne, bo deptało się aż miło. Potem dostałem dwa takie dydki w~łapę i~to samo. Tu poczułem się jak w~domu, bo okazało się, że 100 godzin wygranych w~Call of Duty może przydać się w~życiu codziennym. Na razie sobie radziłem, choć cały czas wisiała na mnie presja ósmego pasażera Nostromo, który pragnie wyjść na świat (tylko mu się drogi pojebały). No i~po tym nastąpił Armagedon i~seria coraz poważniejszych wpadek. Po badaniach osobno dla rąk i~osobno dla nóg nastąpiło badanie wszystkiego razem. Zielone z~dźwiękiem to prawa ręka, jak samo zielone to noga itp. Noż w~mordę jeża, ja tu walczę ze sobą i~się spinam jak mogę, a~każą mi wymachiwać nogami i~rękami! 

---~Coś pan spięty, panie Piotrze ---~powiedziała, a~mój nosiciel odjebał taką rzecz, że padłem ze śmiechu. Otóż czasem w~życiu faceta są takie chwile, kiedy jebnie coś bez zastanowienia, a~jak sobie to uświadomi, to jest za późno.

---~Ale tylko w~niektórych miejscach ---~odpowiedziałem puszczając oko. Mimo, że próbowała zakryć twarz swoją teczką, widziałem, że się uśmiecha i~czerwieni. I~to niby my myślimy tylko o~jednym, co? 

Postanowiłem więcej mordy nie otwierać, tylko skupić się na zadaniu balansując między zadaniem, a~zaciskaniem. Po tym i~kilku innych badaniach poszliśmy do jej biura. 

---~Nogi w~normie, pole widzenia też, określanie odległości też, no i~muszę przyznać, że ma pan naprawdę zręczne palce.

I to jest ten moment, kiedy znów nie zdążyłem się ugryźć w~język i~wypaliłem luzackim tonem 

---~Wiem, znajome też mi to mówią. Kobita się uśmiechnęła szeroko a~jej oczy mówiły: „Taaak, z~takim ryjem, na pewno…” Potem wypisała co potrzeba, ja jej życzyłem dobrego dnia, ona mi udanego wieczoru zawadiacko mrugając okiem. Wychodzę z~gabinetu i~szukam kibla. Znalazłem. Zamknięte. Kurwa. No ale plus był taki, że trochę mi zelżało, więc myślę, że wytrzymam do chaty. Wsiadam w~moje Ferrari (a przynajmniej tak mam napisane na masce, paluchem w~kurzu) i~wio! Poganiam swoje 60 koni mechanicznych. 

Już prawie w~domu i~jak mnie nie naciśnie, to tak jeszcze nie miałem. Walczę ze sobą, z~debilem w~Toyocie, który mi zajechał drogę i~z telefonem, który nagle zaczyna dzwonić. Chuj, cel jest teraz jeden, reszta niech wypierdala. Wbijam do mieszkania, już mnie skurcze w~dupę łapią, a~ja wiem, że to ostatnie sekundy wytrzymałości i~JEST, ZWYCIĘSTWO! Ja pierdolę, jaką ja czułem wtedy radość i~ogarniająca mnie ekstazę to sobie nawet nie wyobrażacie. Człowiek zapierdala za pieniędzmi, za rodziną i~innymi rzeczami by osiągnąć szczęście, a~to się okazuje, że wcale dużo nie potrzeba, by je dosięgnąć… To było wspomnienie tego całego BQP, czy jak on tam się podpisywał w~Internecie. Miał… mam na imię Piotr. Tego biednego frajera cieszyło to, że mógł się wysrać. A~co zadowoliłoby mnie? Śmierć. Taka prawdziwa i~ostateczna. Cóż, może to życie będzie ostatnie? Może siła, która mną kieruje odpuści tym razem? Może zniknie tak samo, jak się pojawiła?

Pracuję więc na magazynie. Magazyn jest ogromny i~znajdują się tam najróżniejsze rzeczy. Od zniczy przez słodycze, aż do sprzętów RTV. Ot, magazynujemy chyba wszystko, co zechcą klienci. Pracuje tam kilka osób. Obecnie mogę skojarzyć tylko Młodego i~Mariusza… Kim oni są do cholery? Pora poszukać odpowiedniego wspomnienia.

Pamiętam, jak przyszedł do nas nowy pracownik. Młody trafił pod opiekę „wózkowego” Mariusza, który najlepiej, jakby sam dostał opiekę. W~domu bez klamek. Nasz Mariuszek to taki gość, który przychodzi, drze mordę na wszystko i~wszystkich, mimo że sam jest obibokiem jakich mało. Niby taki chojrak, że na wózkach to by jeździł jak Van Dame na ciężarówkach, a~wystarczy jeden tekst, żeby cwaniaka spacyfikować. Zresztą to taki kurdupel, że gdyby go postawić na tych wózkach, to tarłyby o~siebie lusterkami. No więc Mariuszek próbuje pokazać Młodemu magazyn, przedstawić ludzi i~przy okazji poszpanować jaki to on wspaniały i~jak to inni muszą się z~nim liczyć. Zamiast go bezpośrednio wkurwiać postanowiliśmy go wraz z~Czupryną (z tego co potrafię wyłuskać ze wspomnień to tylko to, że ten cały Czupryna pracuje ze mną na zmianie i~jest całkowicie łysy), że zabawimy się z~nim. Przyjęliśmy rolę uniżonych względem jego majestatu, co szybko podchwycili inni. Tak więc, jak coś chciał to standardowa odpowiedź „spierdalaj” na czas zabawy została zmieniona na „tak, Mariusz, już się robi”. Wszystko robione teatralnie z~przesadą, Młody zaczyna chwytać, że to jaja, ale nie Mariusz. Mariusz lśnił, jak słońce w~zaćmieniu, duma go rozpierała tak bardzo, że chwilowo przestał się garbić, co dodało jego majestatowi jakieś 4 centymetry. Ludzie polani, sami przychodzili prosząc go o~jakieś zadanie (a to żaden kierownik ani nic, zwykły robol jak my). Mariuszek poczuł się tak pewnie, że wpadł na genialny w~jego mniemaniu pomysł. Otóż, zaczął przedstawiać ludzi wierszykiem. Mickiewicz jak chuj się znalazł. Przy nim oryginał to jak Słońce do Słońca Peru. Wierszyki miał tak żałosne, że ludziom brakowało komentarza. Podchodzi więc do mnie, wskazuję ręką i~mówi Młodemu:

\begin{itquote}
Oto Piotr,\\
Niezły z~niego łotr\\
Kiedyś pod paletę wpodł\\
i tam społ.\\
\end{itquote}

Gdyby to usłyszał Słowacki, to pewnie skończyłaby się jego tęsknota za krajem. Niestety Mariuszek trafił na gościa, który był miłośnikiem tak zwanych Wielkich Bitew, przez co niewiele się zastanawiając, zaraz gdy skończył, poleciałem po bandzie:

\begin{itquote}
A oto Mariusz\\
Wielki prawiczek\\
Gdyby krowy nosiły staniki\\
To by nie widział cycek.\\
\end{itquote}

Nastąpiła chwila ciszy, podczas której uświadomiłem sobie, że przez przypadek kopnąłem Mariuszka w~czułe miejsce i~to dość mocno, bo jego morda zrobiła się biała niczym najpopularniejsza flaga Francji, a~co najlepsze oznaczała dokładnie to samo. Ludzie wokół zaczęli rżeć ze śmiechu, na co Mariusz zareagował fenomenalną grą aktorską, tworząc idealną kreację ryby wyciągniętej z~wody. Koniecznie chciał się jakoś zrewanżować, jednak grupa jego szarych komórek odmówiła posłuszeństwa, poza jedną. Tą w~kieszeni, która dzwoniąc okazała się idealnym pretekstem, by opuścić naszą małą lożę szyderców. 

Gościu miał już przejebane do końca dnia. Pojawiły się nawet pomysły, żeby mu na drugi dzień (czyli dziś) podrzucić jakiegoś playboya albo coś, ale moim zdaniem to by było słabe i~nie wziąłem udziału tym przedsięwzięciu. Cienko wyszedł na tym Młody, który śmiał się razem z~nami, przez co Mariuszek już po pierwszym dniu stwierdził Kierownicy (Kierownica to nasza kierowniczka działu. Cholera, jak ona miała na imię?), że chłopak się nie nadaje. Nie potrafił podać konkretnej przyczyny, po prostu się nie nadaje i~chuj. Kierownica jednak wiedziała, co się stało na magazynie i~Młodego przydzieliła mnie. Dobra, znam więc niektóre persony tego dramatu.

Przyjechałem więc do pracy. Magazyn zamknięty. Wyciągam telefon z~kieszeni. MyPhone Hummer i~patrzę na datę. Gdyby ten telefon mógł mówić powiedziałby pewnie „sobota, ty głupi skurwysynu!”. Sobota, 14 października roku 2014. Znowu cofnięcie w~czasie. Ale to było piękne cofnięcie. Przeżyłem 18 lat w~tym stuleciu, w~moim prawowitym, pierwszym życiu, a~przez rok pewnie za dużo się nie zmieniło. Znam te czasy, więc nie powinno być trudno się odnaleźć. Schowałem telefon do kieszeni. Telefon… To była cegła. Gdyby kogoś nim rzucić to trup na miejscu. Cóż, stara Nokia też była niezniszczalna, a~nie była tak ciężka i~nie była tak uboga, jak ten dużo nowszy telefon. Postanowiłem wrócić do domu. Pora przypomnieć sobie co było z~tym agentem. Ostatnie co pamiętam, to szkolenie w~jakimś tajnym ośrodku kilka miesięcy po rekrutacji. Co było dalej?

Tajny ośrodek -- kilka wcieleń wcześniej.

Siedziałem w~sali konferencyjnej. Nie wiem po co potrzebne było aż tyle miejsca, skoro zebrało się nas tylko dwadzieścia osób. Zaczynało jednak więcej. Ile dokładnie? Nie wiem, jest to tajna informacja. Szkolono nas w~najróżniejszych stylach walki ręcznej, z~zakresu broni białej, by zakończyć na broni palnej. Po co? Dlaczego? Co wymagało od nas takiej perfekcji? Miałem nadzieję, że odpowiedzi na te pytania padną za kilka chwil. Odruchowo spojrzałem na zegarek. Westchnąłem głośno patrząc na nadgarstek, na którym nie było niczego. Zegarki zostały nam wzięte gdy po nas przyjechali.

Byłem w~domu. Dopiero skończyłem prysznic, gdy wróciłem z~codziennych zajęć w~akademii FBI, gdy ktoś zastukał do drzwi. Otworzyłem je, a~do środka bez żadnego słowa weszło trzech mężczyzn. Jeden natychmiast przystawił mi pistolet do głowy i~powiedział, że zabiera mnie w~miejsce, w~którym wszystko się wyjaśni, a~teraz mam się nie odzywać. Zrobiłem, jak kazał. Domyślałem się, że to ludzie FBI. Pewnie chcieli zobaczyć jak działam w~stresie, albo testowali mnie w~inny sposób. Kazano mi zdjąć zegarek i~oddać telefon komórkowy, bo „dwadzieścia metrów pod ziemią nie będzie mi potrzebny”. To było ostatnie co usłyszałem, zanim potężny cios pogrążył mnie w~ciemności.

Teraz siedziałem w~sali wraz z~dwudziestoma zdezorientowanymi ludźmi. Byli to zarówno mężczyźni jak i~kobiety. Byliśmy spokojni, będąc raczej pewni, że to miejsce należące do agencji. Nie wiedzieliśmy jednak, dlaczego się tu znaleźliśmy i~dlaczego w~taki sposób. Nasze gdybania przerwał świst, jaki dobył się z~rozmieszczonych na ścianach głośników. Po chwili zaczęły z~nich płynąć słowa. Słowa, które miały zmienić wszystko. Słowa, które wprawiły mnie w~przerażenie, mimo, że od kilku żyć nic nie robiło na mnie wrażenia.

---~Witajcie! ---~powiedział głos. ---~Znaleźliście się tutaj, ponieważ…

---~Piotr, wszystko gra? ---~moje rozważania przerwał znajomy mi głos. Może nie do końca „mi” co mojemu nosicielowi, którego wspomnienia przyjmowałem coraz szybciej. Należał do Weroniki, „mojej” narzeczonej. Jej głos był ciepły, troskliwy, z~nutą zdziwienia. Cóż, gdybym ja wrócił po pracy i~zobaczył swoją drugą połówkę siedzącą nieruchomo przy stole z~zamkniętymi oczyma, na dodatek nie reagującej na krzątaninę wokół niej, też pewnie byłbym lekko zaniepokojony. Gdybym prowadził zwykłe, śmiertelne życie.

---~Hej, skarbie, wszystko gra. Po prostu miałem dzisiaj ciekawy dzień i~chciałem chwilę posiedzieć w~ciszy. Widocznie odpłynąłem na chwilę ---~odpowiedziałem jej, po czym streściłem to, co mnie spotkało.

---~Głuptasie, przecież wczoraj sam się cieszyłeś, że będzie sobota i~będziesz leżał w~łóżku cały dzień.

---~Co zrobisz, takiego durnia sobie wybrałaś ---~odpowiedziałem, uśmiechając się do niej. Popatrzyła na mnie, westchnęła głęboko przechylając delikatnie głowę. Popatrzyła na mnie i~uśmiechnęła się. Jasny gwint, ten widok oczarował mnie tak bardzo, że mógłbym mieć ten obraz przed oczami całe wieki. 

Miała około metr siedemdziesiąt wzrostu, perfekcyjną talie, piersi nie duże, ale też nie małe. Idealne. Do tego piękne, rozpuszczone włosy. Jakby tego było mało, że była śliczną brunetką z~pięknym ciałem, to zwieńczeniem jej cudowności była śliczna twarz. Piękne, niebieskie oczy znajdowały się nad małym, troszkę zadziornym nosem. Jej usta były wąskie, a~po ich lewej stronie znajdował się pieprzyk. Nie można też zapomnieć o~jej perłowych zębach, które mogłyby być gwiazdą nie jednej reklamy pasty.

Podeszła do mnie, pocałowała w~czoło, a~jej wilgotne, delikatne usta wyzwoliły we mnie pożądanie. Pocałowałem ją namiętnie w~usta, co spotkało się jej aprobatą i~odpowiedzią w~postaci języka szukającego moich warg. Zacząłem ją rozbierać, a~wspomnienia z~tajnej bazy zeszły na dalszy plan.

---~Witajcie! ---~powiedział głos. ---~Znaleźliście się tutaj ponieważ nie ogranicza was umysł. Testowaliśmy was, sprawdzaliśmy. Nie zdawaliście sobie z~tego sprawy, jednak te wszystkie treningi, ćwiczenia i~nauka miały cel. Otóż żaden człowiek nie mógł się tego nauczyć. Mogliście to zrobić tylko i~wyłącznie wy. Żeby zdać testy musieliście korzystać z~doświadczenia zdobytego w~innych ciałach. To pozwoliło nam na odsianie zwykłych śmiertelników.

---~Jest nas więcej? ---~zapytał ktoś ze zgromadzonych.

---~Tak, jest więcej takich ludzi. Jednak ukrywają się, przez co nie idzie podać konkretnej liczby posiadających ten dar.

---~Dar? To przekleństwo! ---~krzyknął ktoś inny.

---~Dzięki nam będziecie mogli zmieniać świat.

---~Jak? ---~tym razem to ja zadałem pytanie.

---~Czas na pytania jeszcze nadejdzie. Teraz jednak podzielimy się wiedzą, którą posiadamy. Być może to, co za chwilę usłyszycie, da wam odpowiedź na nurtujące was pytania. Albo jeszcze więcej pytań.

---~Jesteśmy tajną rządową grupą ---~kontynuował. ---~Nie reprezentujemy jednak żadnego rządu. Nie interesuje nas władza nad konkretnym państwem. Interesuje nas władza nad nimi wszystkimi. Nie jesteśmy jednak dyktatorami. Chcemy pokoju. Osiągamy to jednak przelewając krew. Nie boimy się śmierci, więc mamy prawie niczym nieograniczone zdolności. Naszym interesom przeszkadzał Kennedy, więc jeden z~naszych agentów go uciszył. Zamiast jednak popełnić od razu samobójstwo, postanowił uciekać. Chyba za bardzo podobało mu się życie, jakie prowadził. Cóż, dziś Oswald pewnie żałuje, że tego nie zrobił. To tylko jeden z~przykładów. Możemy więc manipulować historią. Znamy ją, więc robimy listę celów, jakie należy z~niej usunąć. Gdy któryś z~naszych agentów trafi na początek XX wieku, pozbędziemy się Hitlera, Stalina i~II Wojna Światowa nigdy się nie wydarzy! Czy wiecie, że na początku XXI wieku rozpoczęła się Wielka Wojna Atomowa? Nie, ponieważ jeden z~agentów, który trafił w~czasy poprzedzające ogólnoświatowy zrzut bomb atomowych, wiedząc co stanie się przyszłości, którą obserwował w~ciele innego człowieka, zażegnał kryzys. We wrześniu 2001 roku było na świecie dużo punktów zapalnych. Pierwsze bomby zrzuciły USA na Chiny po konferencji, na której Chiny ogłosiły, że kontrolują większą część gospodarki Stanów Zjednoczonych i~najechały je zbrojnie. Stany spróbowały zniszczyć Chiny wysyłając pociski balistyczne z~głowicami nuklearnymi na ich kraj, lecz ci mimo zniszczeń odpowiedzieli własnymi głowicami. Połowa ludzkości wyparowała wtedy w~ciągu kilku godzin. Jedyne, co wtedy wymyśliliśmy to zburzenie wież WTC, gdy agenci odżyją w~czasach sprzed wojny i~zrzucenie winy na Bin Ladena, który także był naszym agentem. Piloci również nie byli przypadkowi. Zabiliśmy kilka tysięcy ludzi, jednak dzięki temu zabiegowi powstrzymaliśmy atomowy holokaust ludzkości. Taka była cena pokoju w~XXI wieku. Każdy z~was dostanie listę celów, które porozsiewane są po całym świecie, w~różnych czasach historii człowieka. Gdy traficie w~czasy, w~których znajduje się cel, eliminujecie go. Nawet za cenę życia. Przecież i~tak się odrodzicie. Nigdy więcej się nie spotkamy, a~od was zależeć będzie, czy zechcecie dać ludzkości pokój i~przetrwanie.

---~Korzystacie z~mocy jaką wam i~nam dano. Wiecie coś o~niej? I~jak to jest, że nikt o~was nie wie? Przecież na pewno nie wszyscy agenci okazywali się godni zaufania ---~pytałem dalej. Ich plan był szalony, ale w~jakiś dziwny i~makabryczny sposób logiczny. Chciałem wiedzieć więcej.

---~Nie znamy genezy naszych zdolności. Niektórzy sądzą, że spotkali tajemniczy byt, lecz nie pamiętają, co się wtedy działo. Ich pamięć urywa się kilka sekund po zobaczeniu świetlistej istoty. Kim ona jest i~jaką rolę odgrywa, nie wiemy. Co do tych, którzy nas zdradzili podchodzimy z~dystansem. Gdy opowiadają oni o~ludziach trzymających władzę, stają się w~oczach ludzi szaleńcami. Świry od NPŚ -- Nowego Porządku Świata, tak ich określamy i~dyskredytujemy gdzie się da. Nie robimy im krzywdy, trochę im pomagamy stać się świrami w~oczach ludzi. Ale poza tym nic więcej. Ich śmierć mogłaby pociągnąć kilka pytań, na które ktoś mógłby szukać odpowiedzi. A~tak? Mimo, że mówią prawdę, uważani są za szaleńców. Niektórzy próbują z~nami walczyć. Ale to nie jest możliwe. Nie zabiją wszystkich agentów w~danym czasie.

---~Ale wciąż próbujemy! ---~krzyknął mężczyzna siedzący za mną. ---~Nie należy się wtrącać! Mącicie w~historii! Walczymy o~to, żebyście zostawili świat tak, jak jest. Jeżeli ludzkość chce wyniszczenia, niech ginie. Udawałem jednego z~was, po to by właśnie teraz wszystkich was zabić, byście nie mącili w~tych czasach. A~w~innych też was znajdziemy.

---~Głupcze! To nie ludzie decydują o~swoim losie! W~wojnie, której zapobiegliśmy zginęły miliardy ludzi, którzy nie mieli wyboru! Który los przypieczętowało kilkunastu ludzi na szczycie władzy ---~huknął głos z~głośników. Lecz mężczyzna nie słuchał. Wyjął nóż, rozciął sobie brzuch, z~którego wyciągnął granat. Zawleczka spadła na ziemię. Nie mieliśmy gdzie uciec.

Zerwałem się z~łóżka. Weronika śpiąca na moim ramieniu obudziła się podczas tego niekontrolowanego zrywu.

---~Czy ja za każdym razem muszę trafiać na dziwaków? ---~powiedziała zaspana. Jej rozczochrane włosy leżały na poduszce, a~jej oczy kleiły się jeszcze od snu.

---~Jak to zawsze? ---~zapytałem. Wybełkotała coś zasypiając. Zrozumiałem tylko słowo „josuke”.

Nie wiem, czy dobrze usłyszałem. Choć nie brzmi to nawet jak słowo, z~czymś mi się jednak kojarzy. Pewnie odpowiednie wspomnienie nosiciela zaraz mnie uświadomi. Poczułem głód. Wstałem z~łóżka i~poczłapałem do kuchni. Otworzyłem lodówkę. „O! Serek pleśniowy!” pomyślałem. A~nie, był tylko spleśniały. Ze wspomnień, które mnie nachodziły, zrozumiałem, że Weronika ma pewną wadę. Kupuje najtańsze rzeczy, szczególnie na promocjach. Nie przejdzie obok przecenionego jedzenia, bo „przecież 2 dni do końca daty ważności to dużo”. Część towaru już wtedy nie nadawała się do jedzenia, więc i~tak trzeba było kupić jeszcze raz, tym razem już w~normalnej cenie. Tak więc traciła kasę na coś, co i~tak trzeba było wyrzucić, a~potem kupić ponownie. I~nie wytłumaczysz jej, że to żadna oszczędność. Nie rozumiałem takiego podejścia. Może nie byliśmy bogaci, ale też myszy nie biegały po podłodze. Tym razem postanowiłem sobie zrobić jajecznicę. Zapach smażonych jajek musiał obudzić moją narzeczoną, ponieważ przyszła do kuchni. Wyglądała na niewyspaną. Nawet wtedy była piękna.

---~Miałam znowu dziwny sen ---~powiedziała siadając przy stole. Co jakiś czas opowiadała swojemu narzeczonemu o~swoich snach. A~to o~tym jak była w~starożytnej Japonii, albo o~tym, że była żoną wielkiego aktora. Dziś chciała mi opowiedzieć kolejny swój sen. Skąd one się jej biorą? 

---~Śniło mi się, że mieszkam w~Japonii i~jestem facetem. Ogarniasz? Na imię mam Josuke. ---~Zrobiła chwilę pauzy jakby próbowała ogarnąć myśli. Josuke. Jednak dobrze słyszałem. Widocznie wtedy miała swój sen. Po chwili zaczęła mówić dalej. ---~No i~byłam zabujana w~dziewczynie. Hanoko bodajże. Albo Hanako. Nieważne. I~ona mnie miała tylko za przyjaciela. Takie głupie to jak w~filmie. Pomijając szczegóły, to na końcu ją zabiła matka. Ale mam nasrane w~głowie ---~powiedziała, uśmiechając się do swych myśli.

Ja za to przestałem myśleć. Oczywistość sytuacji uderzyła mnie niczym rozpędzona ciężarówka. Podszedłem do niej od tyłu i~położyłem jej dłoń na ramieniu.

---~Znalazłem cię.
