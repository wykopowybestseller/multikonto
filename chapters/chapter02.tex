\chapter{Życie I}
\chapterAuthor{klik34}

---~Paweł, nie zapomnij zabrać drugiego śniadania.

---~Dobrze mamo. Mam nadzieję, że pamiętasz o~dzisiejszym dniu?

---~Dzisiejszym? A~który dziś mamy?

---~Siedemnasty listopada dwa tysiące trzynastego roku? Mamo! Chyba nie powiesz mi, że zapomniałaś!

---~Oj, już przestań, tylko się droczę. Idź już do tej szkoły, bo się spóźnisz, jednak niespodziankę sobie sam 
zepsułeś.

---~Też mi niespodzianka, każdy normalny człowiek tylko na to czeka.

---~Jeszcze kiedyś za tym zatęsknisz.

\paraSep

Pogoda potrafi być nieprzewidywalna, jak na tą porę roku było podejrzanie ciepło. Gdyby nie leżący wokół śnieg, można 
by pomyśleć o~późnej jesieni. Temperatura sięgała 10 stopni, przez co śniegi nieco stopniały, odkrywając skrawki płyt 
chodnikowych, pokrytych śliskim lodem.

Pod bramą szkoły zauważyć można było poruszenie na twarzach młodych ludzi. Jakaś myśl zaprzątała im głowę, a~sądząc 
po uśmiechach były to raczej dobre wieści. 

---~Czyżbym o~czymś zapomniał? ---~pomyślałem, po czym szukając w~tłumie znajomej twarzy, nie zwróciłem uwagi na 
skradającą się do mnie niewysoką postać.

---~MAM CIĘ! ---~rozległ się krzyk mężczyzny, a~następnie poczułem na plecach ciężar, który bynajmniej nie był moim 
plecakiem.

Niewysoką osobą był nikt inny, jak mój najlepszy przyjaciel, Szymon. Znaliśmy się chyba od zawsze, a~przynajmniej 
takie miałem wrażenie. Wiedziałem o~nim wszystko, za wyjątkiem jednej rzeczy -- jak się poznaliśmy. Wspólne wspinanie 
na drzewa, zabawy resorakami czy też budowanie ogromnych zamków z~piasku, wydaje się jakby było wczoraj, a~ta jedna 
rzecz, wciąż jak przez mgłę, niezależnie od tego, jak bardzo próbowałem sobie to przypomnieć. Zawsze sobie 
powtarzałem, że byłem zbyt mały, by to pamiętać. Z~Szymonem można było zrobić absolutnie wszystko. Choćbyś miał 
najbardziej szalony pomysł, pełen uśmiechu zawsze się zgadzał. W~razie problemów nigdy nie zrzucił na mnie winy. 
Można na nim polegać jak mało na kim, choć nie można ukryć, ile razy źle się to kończyło. Zawsze jednak pozostawały 
wspomnienia, te wesołe jak i~te mniej.

---~Musisz tak robić za każdym razem? Zawału kiedyś przez ciebie dostanę. Wiesz może, o~co tutaj chodzi?

---~Też się cieszę, że cię widzę. To nie słyszałeś? Dziś jedziemy na całodniową wycieczkę, a~to oznacza dobrze wiesz 
co. Wolne od zajęć! ---~jak zawsze radośnie oznajmił Szymon.

---~Ta, wolne. Znasz przecież naszą wychowawczynię. Zawsze wymyśli jakiś sposób „żebyśmy tylko nie stracili cennych 
lekcji”.

---~Oj przestań, jedziemy dziś do tego miasteczka, o~którym oglądaliśmy film w~zeszły wtorek.

---~No nie gadaj! Ten, gdzie mieszkańcy próbują żyć jak w~średniowieczu!? ---~podekscytowany niemal zapiszczałem jak 
dziewczynka. Od zawsze historia była moją pasją.

---~Dokładnie ten. Dobra, idziemy do autokaru, tam po lewej widzę naszą klasę.

Podróż zakończyła się błyskawicznie. Pochłonięci rozmową, nawet nie zwróciliśmy uwagi, kiedy minęły dwie godziny 
jazdy. Jeszcze chwilę temu otaczała nas betonowa dżungla, a~teraz byliśmy w~centrum wioski jak z~XIV wieku. Słomiane 
domy, zbudowane na zasadzie owalu okrążały „dzielnicę handlową”, a~na samym środku stała studnia. Nie mieliśmy 
przydzielonych zadań. Mogliśmy robić co tylko nam się podoba, lecz nie oddalając się zbytnio od grupy. Niby dziećmi 
już nie jesteśmy, ale nikt nie wie co tam siedzi w~tych młodych głowach. Ja z~Szymonem postanowiłem zobaczyć to, co 
każdy chłopak lubi najbardziej -- pracownię kowala. Dotarliśmy na miejsce po trzech minutach marszu, w~powietrzu 
wyczuwalny był zapach szybko chłodzonej stali. Kuźnia była jedynym miejscem, gdzie nie znalazł się ani jeden płatek 
śniegu. Temperatura była na to za wysoka. Wymusiło to też oczywiście zmianę materiału, z~którego wykonana była sama 
placówka. Kowalem był bardzo uprzejmy, starszy człowiek, który mając tego dnia przerwę, zadowolony oprowadził nas po 
swoim skarbie, pokazując jego wyroby. Pomimo tego, iż wioska to tylko rekonstrukcja, a~sam zawód kowala niemal 
wymarły, jego dzieła naprawdę robiły wrażenie. Lśniące zbroje, podobne do tych, które widzieliśmy w~filmach i~
niewiele gorszy oręż.

---~Chłopcy, a~chcecie poobserwować może łuczników? Ich sprzęt, to też moja działka ---~zaproponował kowal, widząc 
iskierkę w~naszych oczach.

Na wschodzie było miejsce, na które początkowo nie zwróciliśmy uwagi, wyglądało na część wykorzystywaną do zawodów 
łuczniczych lub czystego treningu.

---~Oczywiście, że tak! Tylko wie pan… Z~tamtej linii do celu jest trochę daleko i~nie będziemy widzieć prawie tarczy.

Zawsze można było liczyć na Szymona. Do celów było maksymalnie 50 góra 80 metrów, a~że ślepi nie jesteśmy, cel widać 
byłoby idealnie. Chciał on jednak wykorzystać znajdujący się pięć metrów przed celem i~pięć nad ziemią podest. Kowal 
od razu wykrył intencje, jednak pierwszy raz od dawien dawna otrzymał widownię naprawdę zainteresowaną tym co robi.
Gdy dotarliśmy już na podest, na pozycjach stało sześciu łuczników. Każdy z~nich był gotów wypuścić strzałę w~
dowolnym momencie.

---~Stań tutaj, będziesz idealnie na wprost celu ---~zaproponował Szymon i~ustawił się tuż obok.

Wtedy wypuszczona była pierwsza salwa. Żadna ze strzał nie ominęła tarczy, widać było, że panowie znają się na swojej 
robocie. Cięciwa po chwili została ponownie naciągnięta. Z~emocji nie spojrzałem pod nogi, oblodzone drewno to 
najgorsze, co może was spotkać. Znajdowało się idealnie pode mną. Nogi uciekły mi, momentalnie usuwając grunt pod 
nogami. Spadając, poczułem jedynie ogromny ból, przeszywający moje ciało, dosłownie. Sześć strzał przebiło moje ciało 
w~niemal każdym newralgicznym miejscu. Strzała, która zadała śmiertelną ranę trafiła idealnie w~jabłko Adama. Ktoś 
złośliwy mógłby rzucić żart o~Wilhelmie Tellu. Zdążyłem tylko spojrzeć na Szymona, który był zadziwiająco spokojny 
jak na taką sytuację. Nie minęło kilka sekund, zanim ujrzałem ciemność. Nie spędziłem tego dnia jak planowałem, 
świętując wraz z~rodziną. Na zawsze zapamiętam dzień, w~którym po raz pierwszy skończyłem osiemnaście lat i~początek 
mojego koszmaru…
