\chapter{Życie XIII}
\chapterAuthor{Japer}

Wraz z~bolącym karkiem, który mnie iście piekł, próbowałem ogarnąć otoczenie. Nic nowego, bo skądś w~sumie to znałem. Ciężko nazwać skąd. Mianowicie ciemność. Próbowałem rozpierać się rękoma, żeby zbadać ów teren, lecz stawiał mi opór jakiś szyty materiał. Coś jak sukno, albo inne tekstylia. Wyciągnąłem moje ręce do przodu i~moim oczom ukazał się strumień światła wraz z~widocznym kurzem. Światło było barwy białej, jak zza okna podczas naprawdę angielskiej pogody. Typowa szarówka.

Gdyby wyłącznie bolał mnie ten kark… Zmęczenie ogarniało mnie ciągle. Jakbym nie tylko się nie wyspał. Wręcz czułem błogi bezkres wraz z~przerażającą rozciągłością. Tak, jakbym odczuwał dwa wątki równocześnie. Ciężko skupić się nad jedną myślą, gdy druga jest w~pobliżu, żeby zaćmiła tą pierwszą.

Zresztą jebać to, „spróbuję pchnąć to coś, skoro się rusza” -- myślę. No i~pchnąłem. Moje oczy nie wytrzymały nawału jasności. Wypaliło mi oczy.

\paraSep

Z klasą wyszedłem z~szafy -- mój tułów ruszył się wraz z~tymi drzwiami i~wywalił się na betonową posadzkę, która miała białe wtopione kamyczki. Prawie bym wybił sobie zęby. Próbowałem się czołgać, bo od leżenia, o~ile można nazwać to leżeniem, nie mogłem ruszyć nóg; jakby mi napuchły. Moim oczom ukazał się długi korytarz oświetlony migającymi świetlówkami. W~tle było słychać szum radia wyczerpanego od słabych baterii. Jak się potem przekonałem, było niedaleko mnie ów radio i~wyłączyłem je, żeby nie generowało zbędnego szumu, jak w~drętwym ambiencie. Zdobyłem się na powstanie i~wstałem. Chodziłem niepewnie, gdy napotkałem zarzyganą postać w~kapturze, która opierała się o~filar. Chciałem podejść do tej osoby i~zdjąć kaptur. Lekko dotknąłem to coś i~obsunęło się to na bok, wylewając większy potok wymiocin obok moich butów. Wymioty przypominały fusy z~kawy, okropieństwo. Z~odrazą ściągnąłem kaptur, który niemiłosiernie jebał niczym odyniec po godach, i~moim oczom ukazała się kobieta. Była blada, miała dość pociągłą twarz i~blond włosy zwinięte w~kok z~tyłu. Po czasie zaczęła mi sinieć. Byłem bezradny, nie wiedziałem nawet, jakby jej pomóc. Pewnie i~tak była już martwa. Mnie do martwych nie ciągnie, amatorem kwaśnych jabłek nie jestem.

Idę dalej. Patrzę na pomalowane farbą olejną okna. Postkomunistyczne. Za oknem widzę żółtą łąkę od zeschłych traw. Idę dalej, do innej sekcji budynku. Przed wejściem było widać jakiś napis w~cyrylicy. Litery schodziły z~niego. Naniesione pewnie były na blachę.

Moim oczom ukazały się setki ludzkich ciał. Przestraszyłem się. Było pełno osmolonych łyżeczek, zapalniczek, palników spirytusowych i~brudnych strzykawek wraz z~igłami. Zalatywało mi jakimś masowym zaćpaniem się. Może było im raźniej, ale mi jakoś tak nie pasowało. Poczułem odrazę do tych ludzi. Wręcz się zląkłem, jakbym stawiał czoła uosobieniu strachu. W~tle słyszałem, jakby ktoś puścił kaseciakiem od tyłu jakieś jęki.

Zauważyłem, że ktoś się rusza. Ta osoba powoli wstała, kiedy ja szybko się schowałem za taką glinianą wanienką. Wstałem i~przyglądałem się. Niepotrzebnie. Ten ktoś się do mnie odwrócił. Nie miał kawałków skóry w~oczodołach, wyglądał, jakby oczy miały mu wystrzelić. Był to mężczyzna. Z~grymasem na twarzy podszedł do mnie i~przyjrzał się z~bardzo bliskiego bliska. Czułem odór rozkładającego się jakiegoś gówna; trampek w~gębie razy tysiąc. Byłem, lekko mówiąc, osrany. Wycofał się on ode mnie, a~ja od niego. Potem wrzeszczał, uciekając w~dół. Ja też uciekłem w~przeciwną stronę. On w~schody, ja w~stronę tamtej babki. Usłyszałem potem trzask łamiących się kości i~rzucanego ciała. Na (nie)szczęście się połamał. Schodzę tamtędy schodami. Zakrwawione schody od jego pierdolnięcia, niczym drop u Stanisława Skrileksowicza głową w~stopień. Widziałem wielkie niebieskie drzwi, które dało się otworzyć. Uciekłem stąd.

\paraSep

Na zewnątrz nikogo nie było. Opuszczone wielkopłytowe bloki z~chropowatą fakturą, drzewa, gdzieniegdzie z~liśćmi. Zacząłem biegać. Bieganie było bardzo szybkie, jakbym za jednym krokiem skoczył dziesięć metrów. Przeszedłem się w~stronę jednej z~dłuższej ulic obok tej opuszczonej ćpuńskiej rudery. Widziałem dwóch ludzi. Też biegali, lecz po jakimś czasie jakby spadali z~nóg i~wstawali osowiali. Tacy bezcelowi, bez znaczenia.

Wbiegłem w~uliczkę bardzo wąską, jak drogi do wsi. Ta droga ciągnęła się w~nieskończoność. Domy były symetryczne obok uliczki. Też ciągnęły się w~nieskończoność. Biegłem dalej w~głąb tej uliczki.

\paraSep

Wraz z~biegiem, wchodziłem do domu. Domy dziwnie też się otwierały, tak jakbym tam stał i~robił identycznie to samo. Wbiegłem do domu i~zastałem w~tym domu parę drzwi. Między nimi była żarówka. I~w~sumie białe ściany, chociaż dziwnie mieniły się na różne kolory. Wszedłem do jednych z~drzwi. Zastało mnie coś ciekawego - drzwi nie miały klamek z~drugiej strony. Zadziwił mnie fakt, że znowu widzę tą samą parę drzwi, pomimo przejścia na drugą stronę. Zacząłem biec dalej w~te drzwi i~wchodzić na oślep. Zastało mnie nic. Całkowita pustka, do której wszedłem.

Nagle poczułem się zawieszony w~całym eterze tej nicości. Drzwi znikły, lecz został promień światła padający z~konkretnego miejsca. Słyszę dalsze puszczone jęki, jak w~jakimś psybientcie. I~w~tym czasie nawiedził mnie jakiś byt, który przenikał mnie przez wszystkie kończyny. Czułem się wyjątkowo dobrze. Powtarzał wszystkie moje myśli. Nie mniej jednak, poczułem się dość dziwnie, kiedy przestaliśmy ze sobą rozmawiać, ale przeczuwałem współistnienie z~nim. Poczułem się schizofrenicznie. Potem dźwięk zaczął się podnosić, prawie jak w~przyspieszonym winylu. Potem zapomniałem, co się działo.
