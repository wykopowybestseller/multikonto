\chapter{Życie XII}
\chapterAuthor{Kontestator}

Niestety nie udało mi się przeżyć wybuchu. W~ostatnim momencie przezwyciężyłem instynkt samozachowawczy i~zakryłem 
swoim ciałem Dana. Niestety najprawdopodobniej nigdy się nie dowiem, czy udało mi się go uratować. Ostatecznie nie 
była to najgorsza decyzja, bo ja przecież i~tak nie umieram. W~następne ciało przenosiłem się z~poczuciem 
wypełnionego obowiązku. 

Ponownie jestem zaskoczony moim nowym nosicielem, bo tak postanowiłem nazywać ciała, w~których przebywałem. Tym razem 
było to niemowlę. Początkowo myślałem, że utknąłem w~ciele niezdatnym do samodzielnej egzystencji. Jednak szybko 
przystosowałem się do nowej sytuacji. Miałem szczęście, że trafiłem do normalnej rodziny. Opiekowano się mną 
troskliwie, aż wykazałem pierwsze oznaki samodzielności. Nawet chodzenie okazało się trudne do opanowania w~tym 
młodym i~wątłym ciele. Całe szczęście, że udało mi się zachować jasność umysłu podczas, jak mi się wydawało niezwykle 
długiego okresu dorastania. 

Miałem dostatecznie dużo czasu, żeby ułożyć sobie plan na nadchodzące życie. Zabawnie to brzmi – niemowlę leżące w~
łóżeczku i~wrzeszczące o~zmianę pieluchy czy podanie kolacji, rozmyśla nad swoim życiem. Jednak dokładnie tak to 
wyglądało. Podczas swoich poprzednich wcieleń zdążyłem się przekonać, że nie jestem jedynym człowiekiem obdarzonym 
taką zdolnością, jak pewien rodzaj reinkarnacji. Jednak moim zdaniem użyłem złego określenia. Chyba jednak nie można 
nazwać mnie człowiekiem, istotą ludzką. Normalni ludzie przecież po śmierci idą sobie gdzieś, do nieba, piekła czy w~
inne mniej lub bardziej przyjemne miejsca, których chyba nie będzie dane mi poznać. Ja tymczasem krążę po świecie 
odzierając z~tożsamości dziesiątki ludzi, będąc przy tym niesamowitym nieudacznikiem, który umiera krótko po 
przejęciu nowego ciała. Postanowiłem, że tym razem będzie inaczej. 

Musiałem się skupić nad tym, jak dowiedzieć się, kim tak naprawdę jestem. Długotrwały pobyt w~szpitalu 
psychiatrycznym podczas jednego z~wcieleń, utwierdził mnie w~przekonaniu, że moja przypadłość nie jest typowa dla 
mieszkańców tej planety. Co więcej, przysparza mi ona wrogów. I~nie do końca wiem, czy negatywne nastawienie 
otaczających mnie ludzi jest spowodowane błędami mojego nosiciela z~czasów zanim go zająłem, czy też z~racji moich 
wewnętrznych cech. O~ile w~przypadku Hanako można śmiało powiedzieć, że niechęć matki była zbiegiem okoliczności, to 
sytuacja, w~której się znalazłem zanim poznałem panią psycholog, może już budzić wątpliwości. Poza tymi epizodami 
były także takie, które nie pozostawiły mi cienia złudzeń, że są tacy, którzy polują na podobnych mi osobników. 

Tym razem dostałem niepowtarzalną szansę na odkrycie tego, kim tak naprawdę jestem i~jak wreszcie odnaleźć spokój. 
Drugi raz podczas mojej egzystencji miałem całkowitą kontrolę nad własnym życiem. W~tym życiu postanowiłem zacząć 
rozpracowywać własną tajemnicę. Wiedziałem od czego musiałem rozpocząć swoje dociekania. Musiałem ją znaleźć. Pani 
psycholog, która najpierw postrzeliła mnie, a~następnie popełniła samobójstwo. Chociaż poprawniej byłoby stwierdzić, 
że przeskoczyła do następnego ciała. Jej ostatnie słowa jasno wskazały, że pod tym względem była podobna do mnie. 
Udało się jej dowiedzieć więcej niż mi na temat naszego położenia. Problem tkwił jednak w~tym, że nie wiedziałem czy 
mogę jej zaufać. Nie miałem jednak innego wyjścia niż ją odnaleźć. Musiałem się jednak najpierw do tego porządnie 
przygotować. 

Z uwagi na to, że swoje już przeżyłem, po szkole nie musiałem poświęcać czasu na naukę. Nie licząc zadań domowych, 
które odrabiałem regularnie. Robiłem tak, aby nie ściągać na siebie zbyt wiele uwagi. Z~drugiej strony musiałem się 
też pilnować, żeby nie popisywać się ponadprzeciętną wiedzą. Wychodziło mi to nawet dobrze. Byłem ogólnie uznawany za 
bardzo dobrego ucznia i~udało mi się uniknąć losu 10-letniego geniusza kończącego uniwersytet. Efektem takich działań 
było to, że miałem bardzo dużo wolnego czasu. Pierwszym etapem przygotowań poszukiwań, były regularne ćwiczenia. 
Doszedłem do wniosku, że stąpam po kruchym lodzie i~muszę być przygotowany fizycznie. W~razie konieczności musiałem 
się bronić. Od najmłodszych lat sporo czasu wolnego poświęcałem na treningi sztuk walki. Poza tym sporo czasu 
spędzałem ze znajomymi na podwórku. Gra w~piłkę i~inne zabawy nie stały na przeszkodzie w~przygotowaniu kondycyjnym, a
~pozwoliło mi to uchodzić za w~miarę normalnego. 

Znacznie większym wyzwaniem było odszukanie mojej \emph{bliźniaczki}. Tak zacząłem o~niej myśleć, ponieważ teraz już 
na pewno nazywa się inaczej, chociaż nie musiała nadal być kobietą. Nie mogłem mieć pewności, że znajdowała się w~
ciele osoby żyjącej w~tym samym czasie co ja, ale jakieś dziwne przeczucie podpowiadało mi, że tak jest. Szczęśliwie 
trafiłem do czasów, w~których Internet był już powszechny. Jednak poszukiwania utrudniał fakt, że nie mogłem wprost 
ogłosić kogo szukam. Jeśli tamci wpadliby na nasz trop, prawdopodobnie ponownie by nas rozdzielili, albo nawet w~porę 
zapobiegli ponownemu spotkaniu. Musiałem być bardzo dyskretny. Niestety, w~poprzednim życiu nie zdążyłem jej zbyt 
dobrze poznać. Skąpa ilość wspólnych wspomnień nie ułatwiała zadania. 

Będąc w~liceum, zacząłem prowadzić własnego bloga. Jego tytuł to były ostatnie słowa \emph{bliźniaczki}. 

\begin{itquote}
Obiecaj mi, że mnie znajdziesz!
---~Musisz mnie potem znaleźć.
\end{itquote}

Na szczęście, u nikogo ze znajomych ten tytuł nie wydał się dziwny. Pewnie uznali, że to kolejne z~moich dziwactw, 
którymi nie warto się zajmować. Pomyślałem, że skoro w~ostatnich słowach nie dała mi wskazówek jak ją odnaleźć, to 
może ja powinienem dać jakiś sygnał, aby ona mogła odnaleźć mnie. Szczerze mówiąc, nie spodziewałem się wiadomości od 
niej. To był pierwszy krzyk rozpaczy. 

W związku ze swoim położeniem, zacząłem zajmować się ezoteryką. Zwracałem uwagę na różne aspekty reinkarnacji. 
Jedyne, co udało mi się odnaleźć, to albo niejasne zapiski, albo totalne bzdury. Co prawda, istnieje na świecie wiele 
odłamów religijnych wierzących w~reinkarnację, jednak nigdzie nie znalazłem informacji pożytecznych z~mojego punktu 
widzenia. W~liceum podjąłem desperacką decyzję. Skoro w~otwartym obiegu nie było żadnych interesujących danych, a~
wstąpienie do jakiejś sekty na pewno mi nie pomoże, to pozostało mi tylko jedno -- wstąpienie do służb specjalnych. 
Istnieje mnóstwo teorii spiskowych, że służby takie, zajmują się przeróżnymi dziwnymi sprawami, jak UFO czy 
starożytne artefakty, jak Św. Graal czy inne mityczne skarby. Jedynym wyjściem na sprawdzenie prawdziwości tych 
domniemywań, było zaciągnięcie się do służby i~być może z~czasem przydzielenie do takich zadań, o~ile istnieją 
przeznaczone ku temu komórki. 

Aplikacja na studia policyjne na specjalność bezpieczeństwo wewnętrzne została przyjęta bez najmniejszego problemu. 
Wszystkie testy przeszedłem bez problemu. W~końcu uczyć wielu rzeczy się nie musiałem, a~o~kondycję dbałem już od 
dawna. Same studia nie były niczym nadzwyczajnym. Rygor jak w~wojsku, ścisła hierarchia i~praktyki polegające na 
asyście doświadczonym śledczym. Jednak sprawy, do których nas przydzielano nie były niczym nadzwyczajnym. 
Gangsterskie porachunki czy tropienie seryjnego mordercy. Praca bardzo ciężka i~nie tak satysfakcjonująca jak mogłyby 
to wskazywać filmy sensacyjne. Nie uczestniczyłem w~żadnej strzelaninie czy pościgu, za to naoglądałem się trochę 
gnijących trupów. Jednym słowem -- nic przyjemnego. 

Zmianę sytuacji przyniósł dopiero egzamin końcowy, składany przed specjalną komisją składającą się z~wysokich rangą i~
doświadczonych agentów. W~sumie, to mogłem się tego tylko domyślać po ich wyglądzie i~zachowaniu, bo nazwisk 
egzaminatorów nie poznałem nigdy. Wszystko było owiane aurą tajemnicy i~być może była to część testu -- sprawdzenie 
naszej postawy wobec nieznanych wyższych rangą oficerów. Część pisemna egzaminu nie była niczym nadzwyczajnym. Każdy 
otrzymał opis sytuacji i~miał punkt po punkcie opisać swoje zachowanie. Wszystko to było ćwiczone dziesiątki razy, 
więc ten papierek był w~zasadzie formalnością. Zdecydowanie ciekawsza była część ustna. 

---~Dzień dobry, proszę się przedstawić. Może pan usiąść ---~wysoki, siwiejący już, ale budzący respekt swoją 
posturą, przewodniczący komisji wskazał mi krzesło oddalone około 2 metry od długiego stołu, przy którym siedziała 
pięcioosobowa komisja. 

---~Dzień dobry, Artur Grimm ---~odpowiedziałem siadając na krześle. Zastanawiałem się, po co kazali mi się 
przedstawić, skoro doskonale wiedzą kogo mają przed sobą. 

---~Jest pan z~pochodzenia Włochem? 

---~Nie, moi rodzice urodzili się tutaj. Być może moi wcześniejsi przodkowie stamtąd pochodzą. 

---~Rozumiem, egzamin napisał pan oczywiście wyśmienicie, więc możemy teraz porozmawiać o~konkretach. Opinie pana 
prowadzących są jak najlepsze, dlatego chcielibyśmy jak najszybciej przydzielić pana do pracy.

Tego się nie spodziewałem. Myślałem, że egzamin ustny będzie serią trudnych pytań. Trochę mnie to uspokoiło, jednak 
dalej czułem pewną niepewność. Wszyscy którzy wchodzili do tej sali nie wracali z~powrotem  na ten sam korytarz. 
Prawdopodobnie chcieli dysponować wobec każdego elementem zaskoczenia. Przewodniczący kontynuował. 

---~W~takim układzie chciałbym się dowiedzieć, co pana interesuje. Czy coś spośród rzeczy, które robił pan do tej 
pory wydają się warte uwagi? 

---~Szczerze mówiąc to nie za bardzo. Chciałbym zająć się czymś bardziej nietypowym, nieszablonowym. 

---~Chce pan zostać agentem 007, Jamesem Bondem? ---~komisja wybuchła śmiechem. Widać uznali to za dobry dowcip. Ja 
poczułem się jednak trochę zażenowany. 

---~Przepraszamy, rozumiemy. Spełnia pan wymogi takiej pracy, jednak musi się pan liczyć z~wieloma konsekwencjami 
takiego wyboru. Taka decyzja przyniesie sporo konsekwencji. 

---~Rozumiem, jednak co tak właściwie mi państwo proponujecie? ---~nie wiem skąd nagle wzięło się we mnie tyle 
odwagi, żeby zapytać wprost. 

---~Z~jednej strony nie możemy panu powiedzieć zanim nie wyrazi pan zgody, z~drugiej zaś szczerze mówiąc sami nie do 
końca wiemy. ---~Pomyślałem sobie, że to trochę dziwne -- kierować mnie do pracy, o~której sami nie mają specjalnie 
pojęcia? 

---~No dobrze. W~takim razie jakie są warunki? 

---~W~zasadzie to musi pan umrzeć. ---~Już miałem powiedzieć, że to nic nowego, ale ugryzłem się w~język. ---~praca 
wymagać będzie zmiany tożsamości i~zerwania z~dotychczasowymi nawykami. Oczywiście oficjalnie pana nie uśmiercimy. 
Dostanie pan dyplom i~pracę analityka. Nieoficjalnie natomiast otrzyma pan nową tożsamość i~pana praca zostanie 
ściśle tajna. Rozumie pan tego konsekwencje? 

---~Oczywiście, odpowiadają mi takie warunki ---~w~ten sposób przypieczętowałem swój los. Chociaż nie do końca 
wiedziałem, co będę robić, czułem, że postąpiłem słusznie. 

Budynek w~którym rozpocząłem pracę, niczym się nie wyróżniał. Wyglądał jak dziesiątki innych biurowców w~dużych 
miastach. Różnice pojawiły się dopiero po przekroczeniu drzwi strzeżonych przez recepcjonistkę. Budynek podzielony 
był na strefy oznaczone cyframi od 1 do 5. Aby wejść do każdej z~nich, należało zeskanować swoją kartę dostępową. Na 
mojej wydrukowana została wielka trójka. Jednak nie dane mi było jej użyć ani razu. Byłem prowadzony przez postawnego 
jegomościa przez kolejne korytarze, aż do przestronnego gabinetu. Za biurkiem siedział brunet w~średnim wieku, który 
wnikliwie mi się przyglądał. Odprawił za drzwi mojego towarzysza i~nadal mi się przypatrywał. 

---~Miał pan kiedyś bardzo ciekawe zainteresowania ---~powiedział bardzo pewny siebie. Teraz dopiero uświadomiłem 
sobie, z~kim tak właściwie się zadaję. Ciekawe ile wiedzą na mój temat. Kontynuował:

---~Co pana skłoniło do zainteresowania się tematem reinkarnacji? 

Nie panikowałem. Odpowiedź przyszła mi do głowy bardzo szybko. 

---~Kilka lat temu  będąc na wycieczce w~Niemczech spacerowałem ulicami Monachium. W~pewnym momencie poczułem, jakbym 
znał niektóre zakątki. Zupełnie jakbym bywał tam wielokrotnie. Niektórych miejsc jednak nie mogłem rozpoznać, ale po 
przejściu paru metrów odzyskiwałem orientację. Tak jakbym był tam w~odległej przeszłości. Wrażenie tego jakbym 
pamiętał to miejsce, mimo że nigdy tam nie byłem było tak silne, że postanowiłem poszukać informacji, czy coś takiego 
jest w~ogóle jest możliwe. Bardzo mnie to wtedy zaintrygowało.

W sumie to skłamałem tylko częściowo. Agent nadal mi się przyglądał, ale chyba był zadowolony z~mojej odpowiedzi. 

---~Zatem wygląda na to, że trafił nam się nieintencjonalny. W~międzyczasie postaramy się pana uświadomić, jednak 
póki co, pomoże pan nam w~poszukiwaniach tych, którzy odwrócili się od nas. 

Zrozumiałem, że najprawdopodobniej zbliżyłem się do odnalezienia mojej \emph{bliźniaczki} oraz, że takich jak ja jest 
więcej. Moja sytuacja w~każdym razie się poprawiła, udało mi się zyskać zaufanie agentów i~z ofiary zostać myśliwym. 
