\chapter{Życie VII}
\chapterAuthor{krolka89}

Zaraz, zaraz. Hanako? To czasem nie jest żeńskie imię? Jestem kobietą? Nie sądziłem, że to w~ogóle możliwe. Po tylu przemianach nagle zmieniam płeć? Myśl, szybko.
 
---~Wiem tato. Chyba jednak jej nie skończę ---~odpowiedziałem zmieszany.
---~Mam nadzieję, córeczko. A~teraz wstawaj i~zjedz śniadanie. Mama już usmażyła jajka ---~powiedział cicho ojciec, a~troskę widoczną na jego twarzy zastąpił malutki uśmiech.

Mam chwilę dla siebie. Muszę gdzieś znaleźć lustro. Ile w~ogóle mam lat? Rozglądam się po pokoju. Na ścianach porozwieszane są jakieś plakaty zespołów, o~których nigdy w~życiu nie słyszałem. Jest też mnóstwo pluszowych kotów. O~rany, wszędzie koty. Nawet piżama i~pościel jest ozdobiona tymi sierściuchami. Nie to, żebym miał coś do kotów, ale to one nigdy mnie nie lubiły. Zawsze na mnie prychają i~fuczą, dziwne stworzenia. I~w~tej chwili zauważam zakrwawione bandaże na moich rękach. O~co tu chodzi? Nic nie pamiętam. Na szafie oklejonej zdjęciami widzę lustro. Pochodzę lekko przerażony i~widzę siebie. Mam nie więcej niż 16 lat. Dziewczyna jest nawet ładna. Ciemne, długie włosy, zielone oczy, pociągła twarz, ale też widoczne worki pod oczami i~poszarzała cera. Otwieram szafę i~uderza mnie burza kolorów. Więc chyba jednak dziewczyna nie należy do jakiejś szatańskiej subkultury i~nie tnie się dla zabawy. Chyba czas się ubrać. Ściągam piżamę i~staram się nie patrzeć na moje nowe ciało. Bardzo dziwnie się z~tym czuję, choć muszę przyznać, że dziewczyna jest całkiem niezła, jak na swoją małoletniość. Znajduję jakąś bluzkę z~długimi rękawami, żeby ukryć bandaże. Jest w~żółtym kolorze z~cienkiego materiału, więc nie będzie mi gorąco. Wkładam do niej jasne jeansy. Koło łóżka leżą kapcie. Kapcie w~kształcie kota. To wcielenie, to chyba moje przekleństwo. Ale jestem teraz gotowy, żeby zmierzyć się z~moją nową rodziną.

W międzyczasie mój umysł przesłaniają urywki życia Hanako. Wiem już, że ma starszą siostrę i~obojga rodziców. I~kota. Cholera, czemu ze wszystkich żyjących stworzeń musiało paść na kota? Ale nadal nie wiem, dlaczego mam pocięte nadgarstki.

Kiedy tylko wszedłem do kuchni, mama właśnie z~niej wychodziła. Właściwie wyglądało na to, że ucieka ode mnie. Mimo wszystko śniadanie upłynęło mi całkiem miło. Tata dotrzymywał mi towarzystwa i~czytał przy stole gazetę. Ale było też strasznie cicho. Widocznie takie mają nawyki. Jeszcze nie wszystkie wspomnienia do mnie wróciły, więc czuję się dość dziwnie. Ale zachowanie matki było co najmniej paskudne. Domyślam się, że jest na mnie zła o~to, co zrobiła Hanako. Po śniadaniu poczułem się trochę lepiej, ale mimo wszystko byłem wyczerpany.

---~Dziecko, idź się połóż, bo jesteś strasznie blada. Niedługo przyjdzie pan doktor i~być może pozwoli ci jutro wyjść z~domu ---~powiedział z~troską ojciec.

---~Właśnie o~tym myślałam, bo trochę kręci mi się w~głowie ---~odpowiedziałem i~poszedłem na górę. 

Niedługo po tym ktoś zapukał do mojego pokoju i~bez zaproszenia nacisnął na klamkę. W~drzwiach pojawił się tato i~jakiś starszy facet. Zapewne pan doktor. 

---~I~jak się dzisiaj czujesz, młoda panno? ---~zapytał wesoło lekarz. Nie czekając na odpowiedź zabrał się do odwiązywania bandaży.

---~Dużo lepiej, dziękuję ---~odpowiedziałem, patrząc ciekawie na moje ręce. Nacięć było cztery, wydają się być dość głębokie, ale zagojone. W~teczce doktora zauważyłem moją historię choroby. Ciekawe czy będę mógł ją przeczytać. 

---~Panie Hunh, może pan zostawić nas na chwilę? ---~tata skinął głową i~wyszedł, zamykając za sobą drzwi.

---~Nadal nie chcesz mi powiedzieć, dlaczego to zrobiłaś? ---~zapytał doktor. ---~I~tak będziesz miała obowiązkowe spotkania z~psychologiem, więc pewnie w~końcu się przyznasz. Ale mimo wszystko chciałbym cię zrozumieć ---~powiedział cicho.

---~Ale ja naprawdę niczego nie pamiętam. Nie miałam powodu, żeby się zabić ---~odpowiedziałem pewnym głosem.

---~To może chodziło o~zwrócenie na siebie uwagi? Jesteś nastolatką, a~to się zdarza.

---~Nie, panie doktorze, nie jestem taka. Nie zrobiłabym tego moim rodzicom, a~z~pewnością nie ojcu.

Doktor siedział przez chwilę zamyślony, potem zbadał mi puls, odkaził ranę i~zawiązał czysty bandaż.

---~No dobrze, rana ładnie się goi, myślę że jak tylko poczujesz się na siłach możesz robić co chcesz. Koniec przymusowego siedzenia w~domu. I~uważaj na siebie ---~powiedział lekarz, po czym pogłaskał mnie po głowie i~wyszedł. 

Za drzwiami słyszałam jeszcze przytłumione, męskie głosy. Zapewne ojciec rozmawiał jeszcze z~lekarzem. Po chwili wszedł do mnie do pokoju i~przysiadł na skraju łóżka. Wpatrywał się we mnie przez chwilę i~opuścił wzrok na kolana.

---~Mam nadzieję, że kiedyś mi powiesz dlaczego to zrobiłaś ---~wstał ze zwieszoną głową i~wyszedł.

Zrobiło mi się bardzo przykro, że własna córka potrafiła tak zranić rodziców. A~właściwie ojca, bo widać było jak na dłoni, że z~matką nie miała najlepszych relacji. Zszedłem na dół pooglądać telewizję. Matka krzątała się po kuchni, a~ojciec siedział w~wysłużonym fotelu, zupełnie nie pasującym do wystroju salonu.

---~Sakuro, jedź dzisiaj z~Hanako na zakupy, bo ja muszę popracować w~gabinecie ---~zatrzymał ją ojciec w~progu. Jego ton był surowy i~nieznoszący sprzeciwu. Zupełnie inaczej zwracał się do mnie. Ale właściwie nie dziwię mu się, że tak się zachowuje. Do matki jakby nic nie docierało. Ale na podniesiony ton ojca wstała, zaniosła filiżankę do zlewu i~zaczęła się ubierać. Trochę jak robot, zupełnie bez uczuć. 

---~Ubieraj się, Hanako. Nie mamy całego dnia, a~chciałabym załatwić jeszcze kilka spraw na mieście ---~powiedziała bezbarwnym tonem. Ależ biła od niej szorstkość. Ale zrzuciłem to na wyczyn Hanako. W~końcu, która matka przeszłaby do codzienności zaraz po tym, jak jej dziecko próbowało się zabić? 

---~Już jestem ubrana ---~powiedziałem, zakładając jakieś sandałki. Na szczęście bez obcasów.

---~Ach tak ---~nawet na mnie nie spojrzała i~wyszła.

W samochodzie przyszły do mnie wspomnienia z~dzieciństwa. Zabawy z~siostrą, śmiech mamy, radość ojca. Zupełnie inny obraz tego, co przed chwilą widziałem. Potem doszły jakieś dziwne, niekompletne przebłyski. Burza, hałas, krzyki. Jeszcze głośniejsze krzyki. Sygnał karetki, policji, wrzeszczący tłum. Potrząsnąłem mocno głową, żeby przywrócić się do porządku i~poskładać wszystko do kupy. I~już wiem. Moja starsza siostra zginęła dwa lata temu podczas ratowania jakiegoś małego chłopca z~rwącej rzeki. Udało jej się wyrzucić go na brzeg, ale ją samą porwał nurt i~nie udało jej się wyjść z~tego cało. Cholera, matka nie może patrzeć na moje nowe wcielenie, bo najpierw straciła jedno dziecko, a~teraz mało co nie straciła drugiego? Co ta Hanako w~ogóle sobie myślała? Nie wiem jak długo uda mi się być w~tym wcieleniu, ale czuję wewnątrz, że muszę naprawić wszystko w~tej rodzinie, żeby móc znów usłyszeć śmiech tej kobiety.

Całą drogę do sklepu i~podczas zakupu milczeliśmy. Matka nawet nie wykazała chęci rozmowy. Próbowałem zagaić o~pogodzie, wysokich cenach, ale nic z~tego. Sakuro tylko coś wymruczała w~odpowiedzi. Z~zamyślenia wyrwał mnie znak informujący, że jesteśmy na terenie cmentarza. Wysiadłem za matką i~podążyłem za nią. Kupowała już świeże kwiaty i~znicze. Już po chwili staliśmy nad grobem mojej siostry. Ależ były do siebie podobne! Przypominają mi się zdjęcia na szafie w~mojej sypialni. Byłem pewny, że to zdjęcia Hanako, ale teraz widzę, że poprzyklejała tam zdjęcia swojej siostry. Miesiąc temu minęły dwa lata od śmierci Yoko. 

---~Mamo, czy ty mnie w~ogóle kochasz? ---~odważyłem się zapytać. Matka powoli podniosła głowę i~spojrzała na mnie pustymi oczami. 

---~Oczywiście, kochanie. Skąd ci przyszło do głowy, że może być inaczej? ---~zapytała i~pierwszy raz usłyszałem w~jej głosie jakieś nienazwane uczucie. 

---~Nie przejmuj się mamo, tak mi to przyszło teraz do głowy ---~powiedziałem, chociaż miałem wątpliwości co do szczerości jej wypowiedzi.

Kiedy wróciliśmy do domu, poszedłem od razu do swojego pokoju. Na środku łóżka leżał kot. Chciałem się do niego zbliżyć, ale reakcja była taka jak zwykle -- kot fuknął i~uciekł. Zacząłem przeglądać szuflady i~szafki, żeby móc jak najlepiej wczuć się w~nową rolę. Hanako była zwykłą nastolatką, na szafie faktycznie wisiały zdjęcia jej siostry, było też kilka ujęć przyjaciół i~kota. Po zdjęciach widać, że tworzyli szczęśliwą rodzinę, a~siostry były ze sobą blisko. Telefon na biurku zawibrował. Na ekranie pojawiła się koperta.

\begin{sms}
Tam gdzie zawsze, czekam.
\end{sms}

Nadawcą jest jakiś Josuke. Czuję lekki zawrót głowy i~po chwili już wiem, najlepszy przyjaciel Hanako. Zauważyłem, że wspomnienia teraz przychodzą coraz wolniej. I~boli mnie od nich głowa. Dziwne, bo nigdy wcześniej się tak nie czułem. Podejrzewam, że to przez zmianę płci. Może faktycznie mózgi kobiet i~mężczyzn działają inaczej?

Idę ulicami, które znam, mimo że nigdy wcześniej mnie tu nie było. Chyba nigdy się do tego nie przyzwyczaję. To jakby żyć w~świecie snu. Niby wiesz, że to nie twoje życie i~wspomnienia, ale teraz musisz się dostosować. Nogi same mnie niosą do miejsca spotkania. I~widzę go, siedzi na murku i~gryzie batona. Siadam obok niego, Josuke bez słowa podaje mi baton to ugryzienia. Odgryzam kawałek, bo wiem, że Hanako właśnie tak by postąpiła. Nie chcę wzbudzać podejrzeń, wiem, że i~tak nikt mi nie uwierzy. I~siedzimy chwilę w~ciszy, w~międzyczasie napływają do mnie informacje o~moim towarzyszu. 

Jak ona mogła być taka ślepa? Przecież to od razu widać, że on jest zakochany w~Hanako. Tylko tego mi jeszcze brakowało! Miłość faceta? O~nie, nigdy się na to nie zdobędę. Będę udawał, że tego nie wiem i~może jakoś oddalę się od niego, przestanę się z~nim spotykać i~po prostu zapomnimy o~sobie. Ale też od razu zaczyna mi być go szkoda. Był z~Hanako, kiedy zmarła jej siostra, pocieszał i~wspierał. Był z~nią od zawsze. Widzę go w~każdym szczęśliwym momencie życia mojego nowego wcielenia. To było by nie porządku, tak zostawić go nagle bez wyjaśnienia. Cholera, zaczynam mieć rozterki jak prawdziwa nastolatka. 

---~Wiesz, Hanako ---~zaczął cicho Josuke. ---~Tak sobie myślałem o~tym, co mi ostatnio powiedziałaś. Muszę najpierw porozmawiać ze swoimi rodzicami. Ale myślę, że się zgodzą. W~końcu niedługo wakacje i~mamy prawo wyjechać, gdzieś odpocząć po egzaminach. 

Cholera, ale się wpakowałem. I~co, może jeszcze spać razem? Najgorsze jest to, że oprócz wspomnień, przejąłem też uczucia Hanako i~zaczynam naprawdę, ale tak naprawdę lubić tego chłopaka. Jedynym plusem jest to, że lubi go tylko jako przyjaciela. 

---~Jasne, już nie mogę się doczekać ---~skłamałem. I~kiedy już zaczynałem wymyślać milion wymówek, jak się z~tego wykręcić, Josuke przerwał mój bieg myśli.

---~Coś się stało, mała? Jesteś jakaś dziwna dzisiaj. Jeszcze niedawno szalałaś z~radości na myśl o~wycieczce, a~teraz wydajesz się w~ogóle nie zainteresowana tematem.

---~Masz rację, przepraszam. Po prostu  byłam dzisiaj z~mamą na cmentarzu. Była jakaś dziwna, zapytałam ją czy mnie kocha. Gorąco mnie zapewniła, ale wydaje mi się, że zupełnie nieszczerze. Jej oczy ciągle są takie puste. Ciężko ją rozgryźć.

---~Twoja matka bardzo przeżyła śmierć Yoko, sama wiesz o~tym najlepiej. Widocznie jeszcze się z~tym nie pogodziła. A~rodzice od tego są, żeby kochać swoje dzieci, więc na pewno twoja matka cię kocha ---~mówiąc to otoczył mnie ramieniem i~zaznałem dziwnego uczucia. Jakby poczucie bezpieczeństwa, otuchę i~wsparcie. Mimo zakłopotania poczułem się lepiej. 

---~Dziękuję. Może pójdziemy na jakieś lody, kawę? Masz na coś ochotę? ---~zaproponowałem, bo chciałem jakoś uciec od tej bliskości.

---~Kawę? Przecież wiesz, jak bardzo jej nie znoszę ---~zmrużył oczy i~spojrzał na mnie przenikliwie. ---~Naprawdę nie jesteś dziś sobą.

I wpadłem. To wszystko przez to, że jeszcze nie wróciły do mnie wszystkie wspomnienia z~życia Hanako. Zawsze działo się to w~ciągu kilku minut, maksymalnie do godziny. A~teraz? Minął już ranek, jest prawie szósta po południu, a~ja ciągle mam braki. Właściwie to nie wiem chyba nawet połowy. 

---~Mam dzisiaj gorszy dzień, miałam na myśli, że ja chętnie bym napiła się kawy, a~ty możesz zjeść lody ---~próbowałem jakoś wybrnąć z~tej niezręcznej sytuacji.

Zdziwiło mnie to, że w~ogóle nie wspomniał o~mojej niedoszłej misji samobójczej. Widocznie rozumie mnie jak nikt inny. A~może wie? Będę musiał to jakoś delikatnie wybadać.

\paraSep

Następnego dnia obudziłem się dużo wcześniej. O~siódmej byłem już na nogach. Poczłapałem do kuchni, żeby napić się kawy i~spotkałem siedzącego tam już ojca, który czytał poranną gazetę. Nawet ubrany w~kraciasty szlafrok i~śmieszne kapcie wyglądał na poważnego człowieka. 

---~Cześć córeczko, co tak wcześnie dzisiaj? Przecież są wakacje ---~zapytał podnosząc głowę znad gazety i~uśmiechnął się do mnie. Chociaż on w~tej rodzinie jeszcze mnie lubi.

---~Położyłam się wczoraj dużo wcześniej, przez ten ostatni koszmar szybko zrobiłam się śpiąca ---~powiedziałem i~odwzajemniłem uśmiech. Nastawiłem ekspres do kawy, pamiętając, że ulubiona kawa Hanako to latte z~syropem miętowym. To jest ohydne, ale jakoś muszę  ją wypić, bo wydaje mi się, że ojciec jest równie przenikliwy, co mój przyjaciel. Postanowiłem nie popełniać więcej wpadek, bo jeśli się wyda, że Hanako faktycznie nie jest sobą, to nie wyniknie z~tego nic dobrego. Już ja to dobrze wiem.

---~Jadłeś już coś? Ja mam ochotę na jajecznicę. Zjesz ze mną? ---~zapytałem, zerkając do wnętrza lodówki.

---~Chętnie, dziękuję skarbie ---~odpowiedział z~zadowoloną miną. ---~Mama dzisiaj niestety z~nami nie zje, bo boli ją głowa ---~dodał. Ale powiedział to takim tonem, jakby było to normalne. 

Przeszukując wśród wspomnień, nie znalazłam takiego, które pokazywałoby wspólne śniadanie: ja mama i~tata. Owszem, widzę nas przy stole, ale razem z~moją zmarłą siostrą, więc domyślam się, że mama przestała jadać z~nami, odkąd nie ma z~nami Yoko. Chyba wcale nie mam paranoi. Matka z~pewnością ma do mnie jakiś żal. Nawet nie jest w~stanie zjeść ze mną śniadania. To jest przykre nawet dla mnie, Hanako musiała cierpieć katusze, kiedy matka tak się zachowywała. Może dlatego chciała zrobić sobie krzywdę? Zastanawia mnie fakt, dlaczego jeszcze tego nie wiem. Zauważyłem, że jeśli o~czymś myślę, to po chwili wszystko staje się jasne. Wspomnienia przychodzą, jakby na zawołanie. A~to jedno nie, chociaż bardzo często i~bardzo intensywnie o~tym myślę. Pamiętam tylko kolację, którą Hanako zjadła razem z~rodzicami. Po posiłku położyła się do łóżka i~następną rzeczą jaką pamiętam, jest to, że obudziłem się w~jej łóżku, w~jej ciele. 

---~O~czym tak myślisz, córeczko? ---~zapytał mnie zaniepokojony ojciec. Chyba obserwował moje wewnętrzne rozterki.

---~Pamiętasz jak opowiadałam ci o~wycieczce z~Josuke?  Wczoraj z~nim rozmawiałam i~zastanawiamy się, gdzie się wybrać, może masz jakiś pomysł? ---~próbowałem improwizować, żeby odwrócić uwagę ojca.

---~I~to tak cię martwi? ---~miałem rację, ojciec jest bardzo przenikliwym człowiekiem. ---~Ja bym proponował jakieś jezioro, jest bardzo ładna pogoda, ale najpierw musisz wydobrzeć ---~powiedział patrząc na mój owinięty nadgarstek. ---~I~dorosnąć. ---~mruknął tak, że ledwie go usłyszałem. 

---~Jasne, tato. Tylko sobie planujemy ---~patrzyłem mu prosto w~oczy szukając jakichkolwiek oznak zmartwienia, że jego szesnastoletnia córka wybiera się sam na sam z~chłopakiem na wycieczkę. Widać było, że jest w~rozterce, jakby nie chciał się ze mną kłócić, ale pomysł wcale mu się nie podobał. Nie wiadomo skąd pojawiła się matka, która wtrąciła się do naszej rozmowy.

---~Powinniście wyjechać jak najszybciej, bo podobno niedługo zepsuje się pogoda ---~brzmiało to dość niegrzecznie jak na mój gust, jakby chciała się mnie pozbyć. Ojciec spojrzał na nią krzywo, ale nic nie powiedział.

---~Chciałabym, mamo, ale rodzice Josuke jeszcze nie wyrazili zgody ---~powiedziałem zgodnie z~prawdą.

---~To ja zaraz do nich zadzwonię i~wszystko z~nimi ustalę ---~powiedziała i~poszła do salonu, gdzie zostawiła telefon.
Ojciec miał niezadowoloną minę, którą próbował zakryć uśmiechem. Nie wiem czy kiedykolwiek uda mi się rozgryźć to, co się tutaj dzieje. 

Wciąż nie mogę się przyzwyczaić do obecnej sytuacji. Za każdym razem, kiedy wcielałem się w~kogoś nowego, po prostu czułem, że jestem nim. A~teraz żyję jakby obok. Niby życie Hanako to teraz moje życie, ale wciąż jestem sobą. Zastanawiam się, kiedy to się zmieni. Z~rozmyślań wyrwał mnie dźwięk telefonu. Kolejny SMS od Josuke. Czy ona nie ma innych przyjaciół?

\begin{sms}
Załatwione. Rodzice mówili, że dzwoniła twoja mama. Będę musiał jej podziękować. 
\end{sms}

Podziękować za to, że chce się mnie pozbyć z~domu? Pewnie byłaby zadowolona, gdybym zaszedł w~ciąże i~mogła mnie wyrzucić z~domu. Cholera, ciąża. Jestem teraz kobietą. Wciąż nie mogę w~to uwierzyć. Odpisuję.

\begin{sms}
Matka była aż za bardzo zadowolona z~tego wyjazdu. Jakby chciała się mnie pozbyć z~domu.
\end{sms}

No i~proszę, niech Josuke rozgryzie to ze mną. W~końcu od tego są przyjaciele.

\begin{sms}
Zaraz będę.
\end{sms}

O nie, on tu idzie, a~ja mam na sobie wyciągnięty dres i~dziurawą koszulkę! O~rany, dlaczego ja w~ogóle się tym przejmuję? Przecież powinienem mieć to głęboko w~dupie. Powoli naprawdę staję się  nastolatką. Ale i~tak idę do szafy i~zakładam na siebie przyzwoitą bluzkę bez rękawów i~szorty. Patrzę w~lustro.

---~No, całkiem nieźle się prezentuję ---~mówię do siebie i~w tym czasie wchodzi Josuke.

---~Całkiem, całkiem, mała ---~mówi do mnie z~uśmiechem. ---~Nie musisz się dla mnie stroić, we wszystkim pięknie wyglądasz ---~puszcza do mnie oczko i~śmieje się w~głos. 

Zrobiło mi się głupio, bo faktycznie przebrałem się z~jego powodu. Ale próbowałem to ukryć, śmiejąc się razem z~nim.

---~No więc? O~co chodzi z~matką? ---~pyta autentycznie zainteresowany.

---~No bo pomyśl. Która matka tak chętnie puszcza swoją córkę z~chłopakiem za miasto? Ostatnio zachowywała się jak mumia, ale kiedy tylko nadarzyła się okazja, żeby się mnie pozbyć, ożywiła się i~pofrunęła do telefonu.

---~Moi już dali mi wykład o~bezpiecznym seksie. Wyobrażasz to sobie? ---~wywrócił oczami i~uśmiechnął się do mnie. Mój Boże, tylko nie to. Na pewno nie jest to dobry temat do rozmowy.

---~Moi wykładali mi to już kilka lat temu -- powiedziałem, zaraz po tym, kiedy tylko sobie o~tym przypominałem. ---~Ależ to było niezręczne. Współczuję ci ---~i~śmialiśmy się oboje. Dobrze mi było w~jego towarzystwie. Fajny z~niego przyjaciel. Ale byłoby prościej, gdybym całkowicie stał się Hanako, albo gdybym w~moim prawdziwym życiu był gejem.

---~Masz już jakiś pomysł, gdzie pojedziemy? ---~zapytał, nagle poważniejąc.

---~Tata rzucił hasło jezioro i~myślę, że to nie byłoby głupie ---~cholera, jemu naprawdę na tym zależy.  A~ja chętnie bym się gdzieś wyrwał, pomyślał nad tym wszystkim, ale bez jakiegoś cholernego, gejowskiego romansu w~tle. 

Oboje spojrzeliśmy na drzwi, które otworzył tata, niosąc tacę z~sokiem i~ciastkami. Kochany jest. 

---~Dzięki ---~powiedzieliśmy równocześnie. Tata uśmiechnął się do nas. 

---~Ależ nie ma za co, dzieciaki ---~powiedział i~wyszedł, ale drzwi zostawił rozchylone, na co zareagowaliśmy głośnym śmiechem. Wróciliśmy do naszego planowania wycieczki.

Czułem się taki odprężony. Kiedy umarłem po raz pierwszy, byłem w~tym samym wieku co Hanako. Miło powrócić do tamtych beztroskich czasów. Szkoda tylko, że musiałem wrócić do tych czasów jako nastolatka z~burzą hormonów, w~której jej najlepszy przyjaciel jest zakochany od lat. Postaram się wyciągnąć z~tego życia jak najwięcej. Może nawet napiszę kiedyś książkę „Jak zrozumieć kobietę?”. To byłby hit. 

Nagle Josuke spoważniał. 

---~Mała, ty naprawdę nic nie pamiętasz? Wiesz, że mi możesz powiedzieć wszystko ---~powiedział patrząc mi gorąco w~oczy, zapewniając o~szczerości. Więc on też nie wie. Może uda nam się rozgryźć to razem.

---~Nie, naprawdę. Pamiętam tylko, że położyłam się spać, a~potem obudziłam się w~domu z~bandażami na rękach. Cały czas o~tym myślę. Przecież nie zrobiłabym tego mojemu ojcu ---~powiedziałem, oczekując jakichś nowych spostrzeżeń od mojego przyjaciela.

---~Bardzo dziwne. Pamiętam jak leżałaś w~domu nieprzytomna przez dwa dni. Kiedy cię odwiedziłem, twój ojciec był załamany. Powiedział mi tylko, że wzięłaś jakieś świństwo i~podcięłaś sobie żyły w~łazience. Nie mówiłem mu o~wiadomości, którą mi wysłałaś. Możesz mi odpowiedzieć o~co się z~nią pokłóciłaś?

Z nią? Jaka wiadomość? W~końcu jakiś pasujący kawałek układanki. 

---~Mogę zobaczyć tą wiadomość? ---~pytam pełny nadziei. Josuke wyciągnął swojego smartfona z~tylnej kieszeni i~przeszukiwał wiadomości.

---~Mam, popatrz ---~powiedział i~skierował wyświetlacz w~moją stronę.

\begin{sms}
Ależ ja jej nienawidzę! Wypiła za dużo wina i~wykrzyczała mi w~twarz, że przeze mnie ojciec jej już nie kocha!
\end{sms}

---~Rano odpisałem ci, że jak chcesz to do ciebie wpadnę i~pogadamy, ale nie odpisywałaś i~nie odbierałaś, więc zadzwoniłem do twojego ojca i~powiedział mi, że leżysz nieprzytomna w~łóżku, bo chciałaś się zabić. Wypytywał mnie czy nic nie wiem, ale nic mu nie powiedziałem. Zresztą byłem w~takim szoku, że nawet nie wiedziałem co mam mu powiedzieć. I~nie wspomniałem mu o~twoim smsie, bo chciałem najpierw obgadać to z~tobą. A~że wcześniej jakoś nie było okazji i~widocznie nie byłaś w~nastroju na tę rozmowę, więc czekałem, aż będziesz gotowa ---~wyznał Josuke i~wpatrywał się we mnie szukając odpowiedzi na niezadane pytanie. Czy rzeczywiście kłótnia z~matką mogła mieć taki wpływ na Hanako? Jakoś nie wyglądało na to, żeby były ze sobą bardzo blisko i~nie sądzę, że byle kłótnia mogła być tego powodem. 

---~Pamiętam jak przez mgłę, że leżąc w~łóżku, trzymałam telefon ---~coś zaczyna mi świtać w~głowie, widocznie potrzebuję jakichś szczegółów, żebym mógł sobie przypomnieć życie Hanako. ---~Ale potem, zaczęło mi wirować w~głowie i~zasnęłam. Więcej już naprawdę nie pamiętam ---~powiedziałem zamyślony. 

---~Podkradałaś alkohol rodzicom? Sama wiesz, jak reagujesz na używki ---~powiedział z~lekkim uśmiechem, który przypomniał mi, co Hanako wyprawiała po małej dawce alkoholu. Rany, ta dziewczyna nie powinna nawet wąchać procentowych napojów. Dobrze, że Josuke jest jej prawdziwym przyjacielem i~nie wykorzystał wtedy jej stanu.

---~Nie, mama wtedy przygotowała kolacje, rodzice pili wino, ja oczywiście dostałam piwo bezalkoholowe. Nie przypominam sobie, żebym była nawet bardzo zmęczona, co wyjaśniałoby dlaczego tak szybko usnęłam. To wszystko jest bardzo dziwne. 

Porozmawialiśmy chwilę, włączyliśmy jakiś film i~zjedliśmy to, co przyniósł tata. Josuke czasem patrzył na mnie pytająco, albo przyglądał mi się, kiedy myślał, że nie widzę. Wydawało mi się, że albo podejrzewa, że coś jest nie tak z~jego najlepszą przyjaciółką, albo już wie i~szuka dowodów, że jednak nie oszalał. W~końcu, kiedy zrobiło się już późno, tata zaczął się głośno krzątać po domu, co chyba miało oznaczać, że Josuke już powinien wyjść. Pożegnaliśmy się w~drzwiach, przytulił mnie, a~ja podziękowałem mu za wszystko. Bez niego nie przypomniałbym sobie tamtego wieczoru.

Tata podał kolację, jakieś mięso i~surówki. Grzebałem tylko w~talerzu, myśląc cały czas, co się wydarzyło tamtego wieczoru. Patrząc na wspomnienia, jak wyglądały relacje ojca z~córką, to było coś wspaniałego i~nie wierzę, po prostu nie wierzę, że mogłaby umyślnie chcieć go skrzywdzić. Intuicja podpowiada mi, że matka jest w~to zamieszana. Czyżby to kobieca intuicja? Nie wiem czy można temu ufać, ale na pewno wezmę tę opcję pod uwagę, z~racji tego, że nie mam lepszych pomysłów. 

---~Nie smakuje ci? ---~zapytał ojciec znad talerza. 

---~Nie, nie, jest pyszne. Tylko jakoś nie jestem głodna ---~odpowiedziałem. Otrząsnąłem się z~rozmyślań nad niedoszłą śmiercią Hanako. Właściwie teraz moją niedoszłą śmiercią. 

---~Tylko nie mów, że znowu zaczynasz się odchudzać ---~zaśmiał się ojciec, a~ja mu zawtórowałem. W~tym momencie do kuchni weszła matka. Spojrzała na mnie przez ułamek sekundy, ale w~tym spojrzeniu było tyle nienawiści, że aż skurczyłem się w~sobie. I~w~tym momencie uświadomiłem sobie, że matka musiała mieć z~tym coś wspólnego. Będę musiał na nią uważać, jeśli chcę jeszcze dłużej pobyć w~tym ciele. A~muszę przyznać, że spodobało mi się to ciało, a~właściwie życie, które teraz prowadzę, wykluczając oczywiście relacje z~rodzicielką. 

---~Z~czego się tak śmiejecie? ---~zapytała grobowym tonem.

---~Z~niczego takiego, kochanie. Siadaj zjedz z~nami ---~powiedział tata, a~ona ulegle usiadła i~nałożyła sobie trochę surówki na talerz.

---~Ja już więcej nie zjem ---~powiedziałam. ---~Pójdę już do siebie. Dzięki za kolację, tatusiu ---~tata w~tym momencie uśmiechnął się do mnie i~skinął głową, a~w~matce aż się zagotowało. 

To możliwe, żeby żona była zazdrosna o~córkę? Przecież to chore. Idę po schodach i~słyszę przymilny, kobiecy głos, który zupełnie nie pasował do mojej matki. Dziwne, chyba zmieniła taktykę, bo wcześniej z~pewnością się tak nie zachowywała. Chyba zaczynę się cieszyć z~zaplanowanej wycieczki, bo wyrwę się na chwilę z~tej chorej sytuacji. Leżę na łóżku i~sprawdzam telefon. Żadnych wiadomości. Ale muszę napisać Josuke o~moich spostrzeżeniach.

\begin{sms}
Stawiam na matkę. Spojrzała dzisiaj na mnie z~taką nienawiścią, że jestem przekonana, że maczała w~tym palce.
\end{sms}

\begin{sms}
Twoja matka nie jest matką roku, ale żeby chcieć zabić własne dziecko? Czy ty czasem nie masz paranoi?
\end{sms}

\begin{sms}
A jeśli dosypała mi czegoś do picia podczas tamtej kolacji i~chciała upozorować moje samobójstwo? Nie zapominaj, że była pijana jak miś koala.
\end{sms}

\begin{sms}
Jutro to przeanalizujemy, idź już spać, mała. Dobranoc :*
\end{sms}

Matka zabijające swoje dzieci? To chyba tylko w~patologicznych rodzinach. Moja wydaje się być całkiem w~porządku. Ale i~to wygląda na jakąś chorobę psychiczną matki. Internet może okazać się przydatny. Zapytam wujka Google, co to o~tym sądzi. 

Siedzę z~laptopem na kolanach i~myślę od czego zacząć. „Matka chciała mnie zabić”? To chyba nie przejdzie. „Matka jest o~mnie zazdrosna”. Przecież to bez sensu. Odkładam ze złością laptopa. Pewnie i~tak znalazłabym same bzdury. Znalazłabym? Cholera, zaczynam się zamieniać w~Hanako. Ciekawe czy niedługo stracę świadomość samego siebie. Chociaż muszę przyznać, że już dawno nie udało mi się zostać w~czyimś życiu na długo. To jakieś moje przekleństwo, że kiedy tylko się odrodzę, zaraz ginę ponownie. 

Wstaję wypoczęty, jest ósma rano, wszyscy chyba jeszcze śpią. Nawet kot leży na moim łóżku. Czyżby już ode mnie nie uciekał? Wyciągam ostrożnie rękę, ale kot patrzy na mnie nieufnie, więc ją zabieram. Nie chcę mieć do czynienia z~jego pazurami. Ale i~tak muszę przyznać, że to jakiś postęp z~kocią sprawą.  

Ale to domniemane samobójstwo nie daje mi spokoju. Muszę jakoś sprowokować matkę, żeby przekonać się, że nie oszalałem. Jeśli jeszcze raz spojrzy na mnie tak jak wczoraj, będę miał pewność, że to wszystko jej wina. Może naprawdę jest chora psychicznie i~potrzebuje pomocy? Od śmierci mojej siostry, zachowywała się dziwnie. Może wtedy zwariowała?  

Po południu spacerujemy z~Josuke. Jak zwykle on wcina swoje ulubione, bakaliowe lody, a~ja piję mrożoną kawę, na szczęście bez żadnego syropu. Powoli może uda mi się zmienić przyzwyczajenia Hanako tak, żeby nikt nie zauważył. Mój przyjaciel na razie nie podejmuje tematu mojej matki. Widocznie czeka, aż ja zacznę. Ale chwilowo nie chce mi się o~tym rozmawiać, ani nawet myśleć. Kiedy mijamy kolejną lodziarnię, Josuke patrzy na mnie z~uśmiechem od ucha do ucha i~idzie po kolejną porcję. Kolejka jest spora, więc pokazuję mu na migi, że idę do pobliskiego sklepu po zimny napój. I~nagle wszystko dzieje się tak szybko. Słyszę krzyk, potem czuję ogromny ból, po czym unoszę się w~powietrzu i~upadam na beton. Wszystko mnie boli, ludzie się zbiegają. Tłum jest coraz większy. Znad wrzasków przebija się głos Josuke. Słyszę jego przerażenie i~ rozpacz. Podbiega do mnie, a~ja ostatkiem sił próbuję wyszeptać mu to, co zapamiętałem bardzo dobrze. Kierowca samochodu z~najszerszym uśmiechem jaki w~życiu widziałem i~z najbardziej przerażającym wzrokiem, takim pustym i~szalonym, wjeżdża we mnie z~premedytacją i~nawet się nie zatrzymując, ucieka. 

---~To była moja matka, widziałam ---~udało mi się wyszeptać z~trudem. Mam nadzieję, że zrozumiał. A~ja odchodzę w~ciemność.
