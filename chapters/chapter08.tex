\chapter{Życie IX}
\chapterAuthor{fuurikuuri}

Usłyszałem głosy. 

Przytłumione, tak jakbym miał zatkane uszy, a~ktoś obok mnie rozmawiał. Albo szum oceanu. 

Nie, to jednak ludzie. 

Po chwili udało mi się wyróżnić słowa:

---~Panie prezydencie to on. Najlepszy ze wszystkich dotychczas.

---~Mhm, widzę, widzę… Psychika?

---~Spotyka się z~lekarzem, w~jej opinii jest wszystko dobrze ---~powiedział mężczyzna z~zadowoleniem w~glosie.

---~Hmm… Wie pan, panie generale, że nie możemy sobie pozwolić na powtórkę?

---~Tak, tak ---~szybka i~nerwowa odpowiedź ---~oczywiście, panie prezydencie. Mogę zaaranżować spotkanie z~jego 
psychologiem, jeśli…

---~Nie, nie trzeba. W~najgorszym wypadku zrobimy to, co.…

Znowu przyszedł szum oceanu. To przyjemne uczucie. Bardzo.

Przez ostatnie lata, dekady, wieki bałem się każdej nocy, bałem się tego, co do mnie wróci podczas snu. Teraz nie 
muszę. Wydaje mi się, że moje poświęcenie jest warte mojego spokoju ducha.

Obudziłem się. Nie byłem do końca przekonany znaczeniem mojego snu. Może to tylko dźwięki z~radia dochodzące z~pokoju 
pielęgniarki.

Najgorsza część. Otworzyć oczy. Dwa lata już minęły, a~ja wciąż nie mogłem się przyzwyczaić do tego sztucznego 
światła. Usłyszałem kroki. 

---~I~jak dzisiaj było? ---~zapytała rozmazana twarz.

---~Hmm… – nie wiedziałem. Było inaczej. A~może to moje ciało już się przyzwyczaja do rutyny dnia? ---~Wszystko ok.

Podniosłem się i~omiotłem wzrokiem pokój. Świeże kwiaty, neutralne blade ściany, jakiś obraz nad kominkiem, 
pielęgniarka zapisywała jakieś pomiary w~dzienniku. Wszystko wyglądało blado w~świetle sztucznych lamp. 
Ale czego oczekiwać od budynku ukrytego 60 metrów pod ziemią.

---~Do zobaczenia jutro! ---~powiedziałem jej na odchodne, jak ona ma na imię? Nie wiem, tak często się zmieniają…

Jak zwykle zgłodniałem po kroplówce. Nie wiem jaki jest sens dawania mi płynów odżywczych, skoro jestem po nich 
głodny. Ale to nie ja jestem tutaj lekarzem, ba nawet gdybym był to i~tak nie dorastałbym do pięt ludziom, którzy się 
mną opiekują. Najlepsze głowy na świecie. Ciekawe czy są tutaj, bo mogą pomagać tym wszystkim dzieciom, czy USA im 
tak dobrze płaci. 

Nieważne. Oni przynajmniej mogą wyjść, wrócić na powierzchnię, zobaczyć inne twarze. Ja nie. Czasami wracam myślami 
do tamtego dnia. Nie sądziłem, że ktoś mi uwierzy. Kolejne życie mieszało mi się z~poprzednimi, znowu nie wiedziałem 
co ze sobą zrobić. Nie chciałem oddawać się w~ręce zwykłego lekarza, pewnie znowu wylądowałbym w~wariatkowie. Nie, to 
już przerabiałem. Jedyna opcja to wyjazd do USA. Pamiętam dzień, w~którym stałem przed drzwiami wielkiego budynku, w~
którym ludzie zapracowywali się prawie na śmierć, aby rozwiązywać zagadki medyczne i~szukać leków na choroby, które 
zabierają tyle żyć…

Zdziwiła mnie ich reakcja na to, co im powiedziałem. Spodziewałem się szoku, śmiechu, oburzenia, ba, nawet myślałem, 
że ochroniarze zastrzelą mnie na miejscu, albo ktoś zamknie mnie w~pierdlu. Wszystko stało się tak szybko. Zostałem 
przywieziony helikopterem do jakiejś dziwnej bazy po środku niczego. Po drodze podpisywałem jakieś dokumenty, a~
podniecony lekarz tłumaczył mi, że moja krew jest lepsza niż chemioterapia. Zastanawiało mnie zawsze, czemu tak się 
ze mną dzieje. Czemu nie jest mi dane po prostu odejść. Czy to kwestia jakiegoś żartu i~mojej przeklętej duszy? Czy 
te ciała, które mam we władaniu zmieniają się po tym, jak połączymy się w~jedno? 

Szumiało mi w~głowie od tego. Może to, co przeżywałem, wszystko co zostawiłem za sobą i~o czym chciałem zapomnieć 
miało jakiś cel? Może musiałem dojrzeć do decyzji o~oddaniu się w~ręce nauki? Czyżbym w~końcu znalazł sens? Może 
jednak nie jestem skazany na budzenie się w~nowym życiu i~egzystowaniu czekając na nowe. Zastanawiałem się nad tym co 
noc, ale nadszedł moment, w~którym postanowiłem przestać rozkładać wszystko na czynniki pierwsze i~analizować to, co 
się dzieje. Cieszyłem się z~tego, że może w~końcu będę mógł oddać od siebie wystarczająco, aby pokryć wszystkie 
krzywdy, które wyrządziłem na przestrzeni dekad. Wiele się zmieniło od tego czasu. Pracownicy bazy poświęcili mi 
wiele, abym poczuł się tutaj jak najlepiej. Na początku ciężko było mi się przyzwyczaić do myśli zamknięcia, ale 
wiem, że jestem traktowany jak skarb. Nie, nie będzie to przesada. Pamiętam, że pewnego dnia przyjechał ktoś ważny, 
bo poprzedzający tydzień był pełen przygotowań i~wszyscy pocili się ze zdenerwowania. Nie byłem na tym spotkaniu, ale 
okazało się, że maja mi zapewnić wszystko. Nie chciałem nic, miałem już wszystko -- sens w~życiu, ale nalegali.

Popływałem trochę w~basenie i~wróciłem do swojego apartamentu. 16:30. Za pół godziny idę oddać tygodniową dawkę krwi. 
To były najlepsze momenty. Patrzyłem z~radością na napełniający się worek, wiedząc, że z~każdą kroplą pojawia się 
nadzieja dla kogoś chorego…

Co drugi dzień miałem spotkania z~nią. Psycholog nie był mi potrzebny, czułem się wyśmienicie, ale zgadzałem się na 
sesje tylko i~wyłącznie żeby się z~nią spotkać. Obiecałem sobie jedno. Nigdy więcej nie wyznam miłości drugiej 
osobie. Miałem już swoja dawkę romansu w~życiu, a~raczej w~życiach. Złamane serca, rodziny, oglądanie jak ludzie 
odchodzą i~wiesz, że nie odżyją tak jak ty… Widziałem zawód w~oczach ludzi, widziałem jak ich życia przemijają obok 
mnie. Każdy nowy związek, nawet z~ojcem czy matką powodował rozpadanie się mojego serca kawałek po kawałku. 
Wystarczy… Ale pozwoliłem sobie na spotkanie z~nią. Przecież nie musi wiedzieć, a~ja czułem się bezpiecznie tłumiąc w~
sobie chęć zrobienia pierwszego kroku. Tak było dobrze. Tak mi się podobało. Najbardziej podobał mi się jej uśmiech, 
kiedy pukałem do drzwi jej gabinetu. Doskonale wiedziałem, że nie ma innych pacjentów, że nie jest zajęta, ale to był 
taki nasz mały powrót do normalności. Chociażby zapukać do drzwi. 

Tym razem nie zobaczyłem uśmiechu. Była poddenerwowana. 

---~Usiądź ---~wskazała fotel. ---~Jak się czujesz?

---~Ja? Raczej jak ty się czujesz? Pierwszy raz widzę ciebie taką…

---~Zastępują mnie ---~urwała mi w~połowie zdania i~spojrzała ze smutkiem w~sztuczne okno wyświetlające widok na Alpy. 
Nie rozumiałem. Dlaczego? Po co? Czemu?

---~Powiem im, że nie chcę! 

---~Nic to nie da. ---~Dlaczego ona na mnie nie spojrzy? ---~To nie twoja decyzja…

Wydawało mi się ze się waha. Widziałem, że walczy z~myślami. Przygryzła dolną wargę i~w jej oczach pojawiły się łzy. 

---~Spędzisz ze mną ostatni dzień?

---~Tak ---~odpowiedziałem. Wróciły do mnie te najgorsze myśli i~pytania. Kiedy ja będę miał swój ostatni dzień?

Nie miałem zaplanowanych żadnych medycznych zabiegów tego dnia, więc zgodnie z~umowa mój czas należał do mnie. 
Przechadzaliśmy się po sztucznym ogrodzie i~staraliśmy nacieszyć się swoją obecnością na zapas. Po obiedzie 
zaproponowałem pójście do galerii, w~której dla mojej kulturalnej rozrywki trzymane były kopie wielkich dzieł.

---~Nie ---~odpowiedziała z~bólem na twarzy ---~muszę coś ci pokazać.

Zdziwiony poszedłem za nią w~stronę magazynów i~wind dostarczających zapasy z~powierzchni, skręciliśmy w~boczny 
korytarz. Zawsze omijałem te miejsca, bo nie czułem potrzeby ciągłego zwiedzania całego piętra, a~poza tym ta część 
była zimna, surowa. Wszędzie beton, wiszące u sufitu kable. Za bardzo przypominało mi to, że żyję w~bunkrze. 

---~Dokąd idziemy? 

---~Zanim odejdę musisz to zobaczyć.

Użyła swojej karty dostępu i~weszliśmy do jednego z~magazynów. 

\paraSep

---~Nie rozumiem? ---~nie wiedziałem, czego chcę. Nie wiedziałem po co zabrała mnie do miejsca z~którego wysyłane są 
śmieci, pozostałości jedzenia czy inne niepotrzebne rzeczy. 

Otworzyła jedno z~pudełek. Ku mojemu zdziwieniu w~środku było drugie.

---~Co tu się dzieje? 

---~Zamknij drzwi ---~powiedziała szorstko.

Wyciągnęła metalowe pudło ukryte w~drewnianej skrzyni i~podała mi je z~łzami w~oczach. Spojrzałem na etykietę – „TOP 
SECRET”. Nie dziwiło mnie to. Wszystko co się dzieje w~tym bunkrze jest w~najwyższej tajemnicy. Spojrzałem na nią 
pytająco. Zniecierpliwiona obróciła pudło i~pokazała mi druga stronę. 
Widniał tam napis: „Afghanistan -- property of US Army”. Wciąż nie rozumiałem. 

Wyrwała mi pudełko z~rąk i~otworzyła. W~środku były worki z~krwią. Moją krwią. 

---~Niemożliwe ---~czułem, że w~głowie mi szumi, miałem wrażenie ze zaraz zemdleje.

---~Przykro mi ---~odparła gładząc mnie po włosach. ---~Przykro…

---~Ale… Ale ja im powiem. Powiem, że nie chcę. Jak to możliwe?! ---~zacząłem ciężko oddychać nie wiedząc, jak sobie 
poradzić z~napływającą złością ---~Skąd mam wiedzieć ze to prawda? Skąd?! ---~złapałem ja za ramiona i~zacząłem 
potrząsać. 

Nie zareagowała, pozwoliła mi na wyładowanie swojej złości. Opamiętałem się po chwili.

---~Przepraszam.

---~Nie mów im…

---~Dlaczego?

Złapała mnie za rękę i~pociągnęła w~dalszą część korytarza. Zatrzymaliśmy się przy wielkich metalowych drzwiach. 
Spojrzała na mnie ze smutkiem w~oczach i~wpisała kod. To co zobaczyłem za drzwiami przyprawiło mnie o~mdłości. Pod 
światłem sztucznych lamp na metalowym łóżku leżał człowiek. Raczej był to cień człowieka podłączony do ogromnej 
ilości maszyn, rurek i~innych instrumentów. Do jego ręki pompowana była potrójna dawka tej samej kroplówki, którą ja 
sam dostawałem niemal każdego dnia. Podszedłem bliżej i~zobaczyłem zbiornik z~krwią. Kap, kap, kap. Wpatrywałem się 
jak zaczarowany w~powoli napełniający się worek. 

Poczułem jej bliskość i~złapałem ja za rękę. 

---~Przykro mi ---~powiedziała.

Spojrzałem w~oczy tego człowieka. Oczy puste, oczy proszące o~śmierć. 

---~Co ci jest? ---~zapytałem cicho leżący cień człowieka.

Wydało się z~niego dziwne mamrotanie.

---~Nic ci nie powie. Ucięli mu język, żeby go nie połknął…

Nie rozumiałem, a~raczej nie chciałem rozumieć.

---~Był tu przed tobą. Miał dość i~chciał wracać, ale mu nie pozwolili. Jest zbyt cenny ---~słyszałem, jak mówi przez 
łzy. ---~A potem pojawiłeś się ty… Mają nową zabawkę, ale starej nie wyrzucą.

Patrzyłem z~przerażeniem w~oczy tego człowieka. Patrzyłem na to, co może mnie spotkać. 

---~Kto to?! ---~usłyszeliśmy czyjś głos.

Zobaczyłem tę samą pielęgniarkę z~anielską twarzą, która podawała mi kroplówki. Gdy nas zobaczyła z~przerażeniem w~
oczach podbiegła do przycisku i~włączyła alarm.

---~Stać! ---~krzyknęła, po czym skierowała w~naszą stronę broń. Patrzyłem w~lufę sparaliżowany.

---~Uciekamy!

Czułem jak ręka, którą trzymałem zaczyna ciągnąć mnie w~stronę wyjścia. Za nami rozległy się strzały. Wszystko było 
tak surrealistyczne. Wszędzie migały światła alarmu i~w mózg wciskał się wyjący dźwięk. Zaczęliśmy biec betonowymi 
korytarzami, których ściany wyglądały jak pokryte krwią przez obecną dookoła czerwoną poświatę. Słyszałem zbliżające 
się szybkie kroki, w~każdym korytarzu widać było zbliżające się cienie biegnących ludzi.  Czułem się jak zwierzę 
zaganiane przez łowców. 

---~Tutaj! ---~usłyszeliśmy krzyk z~pomieszczenia obok i~skręciliśmy w~kolejny korytarz w~labiryncie. Zobaczyłem jak 
kawałki ścian odlatują od uderzeń pocisków. 

---~Idioto! ---~ktoś krzyknął za naszymi plecami. ---~Nie w~niego!

---~Szybko! ---~powiedziała zadyszana i~ciągnęła mnie w~coraz to głębsze i~ciemniejsze korytarze. Oby jak najdalej od 
nich, oby jak najdalej od czerwonego światła i~dźwięku dziesiątek ciężkich butów uderzających o~zimną, betonową 
posadzkę.

W końcu poczułem szarpnięcie i~wciągnęła mnie w~jakiś schowek, zatrzaskując za sobą drzwi. 

---~Co się dzieje?! ---~zapytałem przerażony. Miałem wrażenie, że serce wyskoczy mi z~piersi. Nie bylem w~stanie się 
uspokoić i~pierwszy raz w~życiu chciałem, żeby to był jeden z~tych snów, od których budziłem się z~krzykiem w~środku 
nocy. 

Spojrzałem na nią. Była uśmiechnięta. To był ten sam uśmiech, który widziałem pukając do drzwi jej gabinetu. 

---~Nie martw się ---~zdziwiony byłem jej spokojem. ---~Obiecaj mi, że mnie znajdziesz. 

Nie rozumiałem, nie rozumiałem niczego. Przecież pomagałem ludziom, pomagałem dzieciom. Co oni zrobili? Dlaczego 
kłamali?

---~Obiecaj mi, że mnie znajdziesz! ---~powiedziała zbliżając swoja twarz do mojej. 

---~Jak mam cię znaleźć? Jak?! 

Usłyszeliśmy zbliżające się kroki i~szczek broni. Zamarłem i~starałem się nie oddychać.
Poczułem coś zimnego na piersi. To lufa pistoletu.
Jej pistoletu. 

---~Musisz mnie potem znaleźć ---~łzy spływały jej po policzku. 

Pociągnęła za spust. Poczułem, jak coś rozlewa mi się po koszuli. Krew. Czerwona i~gorąca. Upadłem na ziemię nie 
wiedząc, co się dzieje. 

Zamknąłem oczy. Uspokoiłem się. Przeżywałem swoją śmierć nie raz. Za każdym razem było tak samo – człowiek ma jeszcze 
chwilę świadomości. Kilka pierwszych razów zmarnowałem tą chwilę na panikę, ale nauczyłem się z~tym godzić. Kap, kap, 
kap. Plama zaczęła teraz rozlewać się na podłogę. 

Usłyszałem kolejny strzał i~upadające obok mnie ciało. Myśli szumiały mi w~głowie. Czy to możliwe, że ona też? 
Prawdopodobnie tak, spotkałem już wcześniej jednego z~nas. Nie był zbyt przyjaźnie nastawiony.

Ostatnie co usłyszałem przed ostatecznym oddechem to otwierające się drzwi. Ujrzałem czarną sylwetkę trzymającą 
telefon.

---~Panie prezydencie, mamy sytuację… --- powiedział ten sam głos, który słyszałem w~swoim dziwnym śnie.  
