\chapter{Epilog}
\chapterAuthor{bonnie\_21}

Przepaść. Odczuwam tylko ciężki napór powietrza. Nie obracam się, nie wiruję, nie czuję swojego, spiętego mięśniami 
ciała. Macham rękoma i~nogami, próbując złapać się czegokolwiek. Czegoś, czego nie mogę dosięgnąć.

Ciemność. Nic nie widzę, słyszę tylko swój przerywany zachłyśnięciami się powietrzem krzyk. Oczy otwieram tak 
szeroko, aby tylko coś zobaczyć, że nawet piekący i~rwący ból podmuchu nie powoduje odruchu zamknięcia powiek.
Strach. Byłem w~wielu sytuacjach, które sprawiły, że bałem się, głównie sam siebie lub tego, czy tym razem śmierć 
zaboli bardziej niż każdy poprzedni raz. Wtedy przypomniałem sobie, że to nie jest mój pierwszy raz w~tym miejscu. 
Jednakże tym razem było już inaczej… umiałem się zatrzymać…

Desperacja. To niewiarygodne, ile myśli na raz kłębi się w~ludzkiej głowie w~chwili kryzysu. Najpierw ogarnia nas 
chaos, rozpaczliwie staramy się złapać jednego wątku, zlepić poszarpane myśli w~jeden sens. Później uspokajamy 
oddech, bierzemy jeden głębszy.

Motywacja. Obiecane spotkanie z~Weroniką, Martą, Julią, jakiekolwiek imię będzie wtedy nosiła, na pewno się odbędzie, 
to tylko kwestia czasu. Nie zapomniałem. Teraz jednak chciałem spisać wszystko, póki kontroluję ten czas. Póki Cię 
pamiętam, wnuku.

\paraSep

Na początku będzie Ci trudno, ale przekazuję Tobie kilka niezmiernie cennych rad. Wiem, że jest początek i~koniec 
wszystkiego, co nas spotyka. Nie każdy nagina rzeczywistość, tak jak mi na to przyzwolono. Nie wiem jak to się 
dzieje. Mogę Ci tylko powiedzieć, że początek nie boli. Mówię o~bólu, bo to jedyne czego się nie wyzbyłem przez te… 
kilkaset, kilka tysięcy lat. Straciłem rachubę. Mimo, że wydaje się, jakby końca miało nie być, on nastąpi. Kraniec 
wszystkiego jest nieunikniony. Nie wystarczy dobrze zacząć. Nas, ludzi, definiuje to jak kończymy jakikolwiek etap. 
Każdy rozdział mojego istnienia zaczynałem w~nieznanych mi okolicznościach. Wymagało to ode mnie przyzwyczajenia się 
od nowa do osób, miejsc, nazw, emocji. Czasem na bardzo krótko, często trwało to lata. Wiele dałbym za możliwość 
zmiany mrowia aspektów mojego życia. Móc przeżyć wiele, BYĆ raz jeszcze. Ciężko mi nazwać oddychanie, patrzenie, 
ocenianie sytuacji, często tylko jednego dnia -- życiem. Nazywam je byciem.

Najpierw pamiętasz swoją matkę, babkę, rodzinę, ale tylko przez kilkaset być. Ich twarze tracą na ostrości. 
Najczęściej myślałem o~matce. Miałem dni, kiedy zastanawiałem się, czy była blondynką, może Azjatką, mieszkała w~
Londynie, śpiewała arie operowe? Nie znam jej imienia, ale kiedyś znałem. Wiedziałem zawsze, że jest. Była. Nie czuję 
tęsknoty. Nie rozumiem nostalgii. Nie umiem kochać z~definicji, czyli tak jak kochają inni ludzie. Jednak jest 
moment, w~którym ze zdwojoną siłą, wrócą Ci te wspomnienia, ale na bardzo krótko, bo za chwilę wpadniesz w~rozpacz… 
aż zapomnisz, z~jakiego powodu jest ci smutno.

Chciałbym mieć kontrolę i~bezpośredni wpływ na to, co miałoby być następne. Albo chociaż częstsze chwile, żeby złapać 
oddech na nowe jutro. Najczęściej umierałem w~cierpieniu i~przenikliwym bólu, ale zdarzało się przejść do następnego 
świata spokojnie. Budziłem się zrelaksowany, z~nadzieją, że tym razem będzie łatwiej. Wiedziałem, że pomimo kolejnej 
szansy, będzie też czekało kolejne niebezpieczeństwo i~możliwość porażki. Ty musisz być ostrożny. Pamiętaj. Uważaj na 
grot strzały, nie daj się złapać niebezpieczeństwu w~garści, w~lustro patrz z~dystansem…

Zostawiam Ci swoje największe demony, to co było mi dane przeżyć z~krytyczną intensywnością. Ostatnia opowieść jest 
moją najkrótszą, ale i~najtrudniejszą. Uwierz, ciężko opisać stan, w~którym się znajduję. Przeczytałeś cały dziennik, 
ale wysłuchaj jeszcze swojej pierwszej historii…

\paraSep

Zanim zaczęła się nieskończenie długa droga ciemnością w~dół, siedziałem w~swoim fotelu ze szklanką szkockiej, jak to 
miałem w~zwyczaju, odkąd wiek nie pozwalał mi na jazdę konną. Od ostatniego czasu, kiedy „budziłem się na nowo” 
minęły wieki. Zmęczyłem się ciągłą walką z~samym sobą i~otaczającym światem. Kiedy zostałem Bud’em, który mieszkał z~
rodziną na ranczu w~Luizjanie, pomyślałem, że to dobry czas na chwilę odpoczynku. Miałem może z~czterdzieści lat, 
kiedy otworzyłem oczy i~zobaczyłem duży, świecki dom na pograniczu stanów. Wiedziałem, że jak tylko mi się znuży 
obecne oblicze, uczciwego i~szanowanego południowca, wystarczy odegrać pijacką burdę z~młodszymi od siebie, 
zacofanymi amerykańskimi kowbojami, a~wtedy zacznie się nowe. Jednak każdego dnia, mówiłem sobie, że jeszcze nie 
dziś. Jest chwila wytchnienia. Nie pamiętam, jak długo odkładałem decyzję o~opuszczeniu tej rodziny, wiedziałem 
jednak, że nigdy nie żyło mi się lepiej. Nad wyraz młoda kobieta, pachnąca świeżym praniem i~pieczywem to dla 
mężczyzny, po tylu boleściach, najlepsza pociecha. Nie kochałem Marion, straciłem już tą zdolność wieki temu. Kiedyś 
potrafiłem przejmować uczucia nosicieli. Nauczyłem się nie przywiązywać do okazywanej mi czułości, ale również 
umiałem oddać ją wzajemnie, aż z~nawiązką. Poznałem wszystkie tajniki miłości fizycznej i~dobrze wiedziałem jak 
zaspokoić kobietę. Marion lubiłem bardzo. Była dobrą kobietą, tak ją możesz zapamiętać. Na końcu tylko to się liczy w~
ludziach. Od dobroci lub zła rozchodzą się kolejne cechy charakteru. Ona była tą dobrą. Miała długie ciemne włosy i~
piwne oczy, jak to południowa kobieta w~Ameryce. Prowadziła rancho na równi ze mną. A~w~weekendy jeździliśmy konno po 
okolicznych białych, błękitnych i~złotych ranczach Luizjany. Doczekaliśmy się nawet trójki dzieci. Ale to wiesz. 
Jednym z~moich dzieci jest Twój ojciec, Danny. W~moim domu, nikt nie miał prawa polować, korzystać z~broni i~bić się, 
nawet kiedy nie patrzyłem. Moje dzieci złamały tą zasadę tylko jeden raz, kiedy Twój ojciec z~wujkiem Roy’em wrócili 
do domu cali w~gniewie, rozemocjonowani, wiedziałem, że poróżniła ich ochota podwędzenia strzelby, którą jeszcze 
poprzednik mojego obecnego ciała, kupił do obrony ziem przez intruzami. Tamtego dnia, strzelba gościła w~tym domu po 
raz ostatni.

Ku zniesmaczeniu starych już, nowych przyjaciół, moja rodzina nigdy nie brała udziału w~corocznych obchodach 
dożynkowych. Z~poprzednich wcieleń, jedną rzecz pamiętałem nad wyraz wyraźnie, pierwszą śmierć. Bałem się łuków. 
Słyszałem naciąganie i~świst z~jakim wystrzeliła strzała prosto w~moją pierś. Kiedy dowiedziałem się, że Danny, 
przywiózł ciebie do mnie na wakacje, nie sądziłem, że będziesz aż tak nieposłuszny. Byłeś wszędobylski… Chciałbym 
powiedzieć, że przypominałeś mi mnie, jak byłem mały, ale nie pamiętam nic więcej z~dzieciństwa niż zimny trójkąt 
wbity w~moje ciało. Kazałeś nazywać się wagabundą i~wracałeś późnym popołudniem ze zdobyczami kolorowego kamienia lub 
ropuchy znad Missisipi.

Ostatniego dnia mojego istnienia siedziałem w~swoim gabinecie, studiując kolejne dzieło na temat Nicola Tesli. Byłem 
już w~podeszłym wieku. Nagrywałem na dyktafon czytane przeze mnie na głos najciekawsze moim zdaniem kawałki z~
życiorysów ulubionego fizyka, którego chciałem rozgryźć. Czułem się wyjątkowy. Miałem moc, której nie miał nikt inny. 
Ja chciałem być tym, który odnajdzie nieodnalezione. Jednak los nagradza i~karze uczciwie. Ja dostałem niezliczoną 
ilość szans. Być może pozostawiłem po sobie na świecie coś podobnego do Tesli. Odkryłem coś, co może być na wieki 
nieodkryte. Pijąc już trzecią szklaneczkę ulubionego trunku zachłysnąłem się nagle. Poczułem ból nie do opisania, 
mimo oswojenia się ze śmiercią, minął długi czas, od kiedy ostatni raz czułem ten ból. Dopiero teraz zauważyłem pewną 
relację między różnymi rodzajami bólu: głowy, płuc, kości i~… śmierci. Bolało tak, jak mnie milion razy, a~innych 
tylko raz w~życiu. Mimo poczucia palącego, zalewającego płuca alkoholu, wziąłem głęboki oddech i~odkaszlnąłem 
wyraziście, raz, drugi, trzeci. To było tylko zachłyśnięcie się napojem. Śmiałem się wtedy głośno. Myśl, że można we 
własnym domu dokonać prawie, że próby samobójczej ulubionym napitkiem, dodała mi sił na przezwyciężenie strachu przed 
zabraniem Cię na tegoroczny festyn dożynkowy. Tak bardzo chciałeś. Po kolejnym kuflu piwa nawet nie zauważyłem, jak 
znikasz mi i~Marion z~oczu. Wszystko działo się za szybko. Zamroczony alkoholem nabrałem niewyobrażalnej odwagi i~
poprosiłem organizatora, który był moim dobrym kolegą o~pozwolenie na wykonanie jednego strzału z~łuku. Gromkie brawa 
przyjaciół, którzy całe ich życie uważali mnie za deklarowanego pacyfistę umocniły mnie w~mojej nieoczekiwanej nawet 
dla mnie decyzji. Jedynie Marion, dobra kobieta, chciała mnie od tego szaleństwa odciągnąć.

Podniosłem łuk. Brawa! Wybrałem strzałę. Czuję pulsującą krew. Brawa! Dotykam grotu, jest ostry. Nakładam strzałę, 
naciągam łuk. Słyszę brawa i~okrzyki! Krzyki!! Nie wycelowałem dobrze, nie wycelowałem w~Ciebie, ale trafiłem w~
Ciebie. Zabiłem Cię, mam tylko nadzieję, że nie do końca śmiertelnie.

\paraSep

Nie wiem, czy nadejdzie dzień, w~którym wrócisz na miejsce swojej pierwszej śmierci, do domu babki i~dziadka. Mam 
natomiast nadzieję, że jesteś taki jak ja, że podróżujesz i~doświadczasz niesamowitości, jaką ja odczuwam. Oby to nie 
była tylko moja fantazja. Opowieści spisane przeze mnie w~tym dzienniku, są częścią mnie, ale być może staną się 
także częścią Ciebie. Nazywam go multikontem. Jeśli jest Ci to dane, dzięki przeczytaniu rękopisu unikniesz błędów, 
jakie ja popełniałem, a~ja będę musiał zapamiętać jedno – żeby tu zaglądać od czasu do czasu i~delektować się lekturą 
przeszłości. Dziennik zostawiam w~moim gabinecie. Upewniłem się pod kątem prawnym, że przez wieki nikt nie ruszy 
domu. Dzisiaj, kiedy znasz tą historię, jestem już spokojny. Zatrzymałem się na chwilę, bo chcę, żebyś przeczytał, iż 
czekam na Ciebie w~Paryżu, pod Wieżą Eiffla. Nie zdziw się, że nie będę sam. Znajdziesz mnie tam zawsze każdego 25 
dnia miesiąca. Obiecałem to także Weronice. Mam nadzieję, że spotkam się z~tobą i~z nią w~nowym, lepszym życiu…
