\chapter{Życie XV}
\chapterAuthor{fl4meee}

Obudziłem się, pot spływał po moim czole, skroniach i~każdym centymetrze skóry, który nie był okryty materiałem. 
Koszulka, która miałem na sobie przykleiła się do mnie, jak gdybym przed chwilą pływał w~niej w~basenie, albo innym 
zbiorniku. Ręce miałem zimne i~spocone. Przez chwilę nie widziałem co się stało, czy to o~czym przed chwilą myślałem 
było w~innym życiu czy może to był sen? Czy to już nowe życie? Czy ja w~ogóle kurwa żyję?! 

Leżałem na łóżku, które było dosyć przyjemnie. W~pomieszczeniu było zbyt ciemno, by cokolwiek zobaczyć, a~jedynym 
źródłem światła była malutka żaróweczka przy włączniku. Tak mi się przynajmniej wydawało. Leżąc jeszcze przez chwilę, 
myślałem co zrobić. Myśli miałem ciężkie, każda z~nich przytłaczała moje ciało, coś na wzór tego płynu, w~którym 
obudziłem się w… którymś z~kolei życiu. Przez chwilę pomyślałem o~tym, że może to wszystko było snem, że w~końcu się 
obudziłem. Tłumaczyłoby to te zimne poty. Czułem się jak wariat, który zdaje sobie sprawę ze swojej choroby. 
Podniosłem tors i~podpierając się rękami wpatrywałem się w~malutkie światełko, znajdujące się na ścianie po lewej 
stronie. Zrzuciłem z~łóżka nogi i~poczułem dywan, był bardzo delikatny i~puszysty. Uczucie to można było porównać do 
stawiania kroków na nagrzanej od słońca trawie. Od razu ruszyłem w~stronę włącznika by zapalić światło. Przecierając 
oczy pokonałem krótką drogę. Położyłem na nim dłoń i~nie myśląc co zobaczę nacisnąłem. Jarzeniówka na suficie zaczęła 
migać, by po chwili rozbłysnąć w~pełni. Moim oczom ukazał się mały pokój z~łóżkiem na środku. W~rogu, naprzeciwko 
łóżka wisiał stary telewizor ale jego wtyczka była urwana, więc w~sumie był tam zbędny. Ściany były szare, w~jednym z~
rogów przy suficie widać było ślady wilgoci. Obok mnie były drzwi z~zasuwką na łańcuszek, a~w~ich zamku znajdował się 
klucz. Po przeciwległej stronie zauważyłem okno zasłonięte roletą. Tuż przy łóżku była mała szafka, na której leżało 
coś białego, jakiś proszek, chyba narkotyki. 

Podciągnąłem roletę i~otworzyłem okno, by zobaczyć gdzie jestem. Było dość ciemno choć nie wiem dokładnie która była 
godzina. Światło księżyca przebijało się przez gęste chmury. Padał deszcz. Spojrzałem w~prawo i~zobaczyłem napis 
„MOTEL”. Hm, zawsze coś -- mruknąłem do siebie. Na wprost było ogrodzenie, a~za nim droga słabo oświetlona jedną, 
jedyną latarnią. Odwróciłem się i~zauważyłem torbę leżącą obok łóżka. Na niej leżały spodnie, bluza, płaszcz i~
wysokie buty. Szybko doskoczyłem do spodni i~torby, by sprawdzić co się tam znajduje. 

Wprawdzie wspomnienia mojego nosiciela powoli do mnie dochodziły, jednak nie miałem pewności, czy to wszystko jest 
prawdą. Zbyt dużo już tego było. Jakoś przestałem sobie ufać.
 
W spodniach znalazłem telefon. W~kontaktach był zapisany tylko jeden numer i~to pod dziwną nazwą „V”. W~skrzynce 
odbiorczej był jeden SMS z~współrzędnymi. Czytając je przypomniało mi się, że miałem w~to miejsce coś dostarczyć, 
chyba jakiś towar.

---~Kurwa ---~powiedziałem ---~przecież szef mnie zabije. 

W tym momencie przypomniało mi się, że przecież śmierć nie jest mi obca, a~wypowiedziane zdanie brzmi śmiesznie.

Kolejne wspomnienia nosiciela uświadamiały mi, że nie jestem przykładnym obywatelem, byłem członkiem czegoś na wzór 
gangu związanego z~narkotykami. Oprócz komórki, w~spodniach były kluczyki do samochodu i~pieniądze, w~torbie 
znalazłem broń i~dużą, na wyczucie -- pięciokilogramową paczkę „czegoś”, co zapewne miałem dostarczyć. 

Przez poznawanie tożsamości nowego nosiciela zapomniałem o~przytłaczających mnie myślach, raczej wszystko wróciło do 
normy. Nie czułem się już jak wariat. Widziałem, że żyję kolejnym życiem, jednak zupełnie innym niż wcześniejsze. To 
życie może być bardziej „kolorowe”, ale i~dużo krótsze niż pozostałe.
 
Ubrałem się, posprzątałem nosem resztki prochów z~szafki, zabrałem co moje i~opuściłem motelowy pokój. Znajdowałem 
się teraz na korytarzu, był długi i~kiepsko oświetlony. Na jednym końcu widać było sprzątaczkę. Starszą, zniszczoną 
ubogim życiem kobietę. Widząc napis „wyjście” po lewej stronie, udałem się w~jego kierunku. Wyszedłem na powietrze 
poprawiając płaszcz. Deszcz nadal padał mocno. Przede mną była stacja benzynowa, a~po prawej parking, na którym stały 
trzy stare samochody. Stawiałem kolejne kroki, nie zwracając uwagi na kałuże. Chciałem się dowiedzieć, gdzie 
dokładnie jestem, więc udałem się na stację. Tuż przed wejściem poczułem, że narkotyki zaczęły działać. Zmysły mi się 
wyostrzyły, stałem się strasznie bystry. W~wejściu powitał mnie uprzejmy głos starszego pana.

---~Dzień dobry, albo i~dobry wieczór; godzina snu dobrze panu zrobiła ---~powiedział.
 
Dotarło do mnie, że jest właścicielem nie tylko stacji ale i~motelu w~którym nocowałem. Jego aparycja zachęcała do 
rozmowy, a~siwe, rzadkie włosy powiewały na wietrze, wytworzonym przez wiatrak ustawiony obok kasy. Wzrok miał 
przenikliwy, oczy świeciły. To chyba przez jego wiek, ale wyglądały trochę tak, jakby miał zaraz się rozpłakać.

---~Ach tak, szybko się regeneruję ---~powiedziałem z~udawanym uśmiechem, bo narkotyki robiły swoje. ---~Skorzystam z~
toalety, jeśli mogę.

---~Oczywiście, do końca regału z~napojami i~w prawo ---~odparł.
 
Udałem się we wskazane miejsce. Toaleta nie wyglądała najlepiej, ale czego spodziewać się po takim miejscu. 
Spojrzałem w~lustro. Zmarszczki znacząco dodawały mi charakteru, miałem dłuższe włosy, z~których spora część była 
siwa. 

Na szyi miałem tatuaże, nie wiem co oznaczały, ale było kilka symboli, w~tym wąż uroboros, co ze względu na to kim 
jestem mnie zaciekawiło. Oprócz tego miałem wytatuowane litery „ZJTS”, Bóg jeden wie od czego był to skrót. W~mojej 
sytuacji przychodziło mi na myśl tylko jedno rozwinięcie „zawsze jest tak samo”. Patrząc głęboko w~moje wielkie, 
rozszerzone źrenice, przypominałem sobie moje wcześniejsze życia. Myśl o~nich sprawiała, że znowu czułem się jak 
wariat. Przypomniałem sobie psycholog, która zastrzeliła mnie i~siebie mówiąc, że muszę ją odnaleźć. Wtedy coś mnie 
trafiło. Tak, to przecież był mój cel, miałem jej szukać! Takich jak ja jest więcej! 

Szybko obmyłem twarz wodą. Wychodząc z~łazienki staruszek chciał jeszcze ze mną porozmawiać, próbował zagaić rozmowę. 
Widać było, że na tym zadupiu rzadko kiedy ma okazji z~kimś pogadać. Nie miałem na to czasu. Rzucając mu kilka 
banknotów za nocleg, widziałem jak jego oczy nabrały jeszcze większego błysku. Dawno chyba nie widział kilku 
banknotów o~nominale większym niż pięćdziesiątka. Zapytałem go jeszcze o~to, gdzie zaparkowałem.

---~Ze zmęczenia zapomniałem, wie pan ---~dodałem.

---~Parking obok stacji, samochód pośrodku ---~wykrztusił, będąc jeszcze w~szoku przez ilość pieniędzy jaką mu 
zostawiłem.
 
Wybiegłem ze stacji, udając się w~kierunku auta. Był to jakiś ford, nie wiem dokładnie co to za model, to nie było 
istotne. Wrzuciłem torbę na tylne siedzenie, a~sam usiadłem za kierownicą. Odpaliłem silnik i… 

Zamarłem, to co przed chwilą miało dla mnie sens, szukanie tej kobiety znowu stało się bez znaczenia, przecież nie 
miałem żadnego śladu, żadnej wiadomości. Nie miałem kurwa niczego! Niczego poza kolejny bezwartościowym życiem które 
zmarnuje wjeżdżając na czerwonym świetle na skrzyżowanie lub wdając się w~strzelaninę z~członkami innego gangu. 
Bezsens. Nicość. Kim ja jestem?! Czym ja jestem?! Zabieram ludziom ich ciała, ich życia by zniszczyć je w~jakiś durny 
sposób! Dla mnie są bezwartościowe, a~dla nich były wszystkim!

Jestem przeklęty, nie chce żyć! Ten koszmar trwa dziesiątki, setki, tysiąc lat a~ja nie mam z~tego nic! Nie widzę w~
tym sensu…

Myśli znowu zaczęły mnie przytłaczać, chyba zaczynałem wariować. Poprzednie życia zaczęły mieszać mi się z~
teraźniejszym. Czułem się jak nastolatka, jak żołnierz, jak prześladowany przez żonę mąż! Czułem się każdym, a~
zarazem nikim. Ja chciałbym być po prostu sobą!

Ciśnienie uderzyło mi do głowy, zrobiło mi się słabo, uderzyłem czołem w~kierownicę. Słyszałem głośny dźwięk 
klaksonu, który z~każdą sekundą był coraz cichszy. Po chwili znowu widziałem ciemność, w~której tkwiłem i~nie mogłem 
się wydostać… Ciemność…
