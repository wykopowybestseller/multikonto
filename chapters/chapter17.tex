\chapter{Alternatywne zakończenie}
\chapterAuthor{Usmiech\_Niebios}

\begin{itquote}
„Nie ma się co ekscytować”\\
uprzejmie złodziej rzekł\\
„Wielu spośród nas uważa\\
Za żart życie oraz śmierć”

\vspace{.5em}

„Lecz ty i~ja minęliśmy pogląd ten\\
nieprzeznaczony on nam\\
nie okłamujmy zatem się\\
powoli kończy się czas”
\footnote{Jimi Hendrix, \emph{All Along The Watchtower}. Tłumaczenie \nick{Usmiech\_Niebios}}
\end{itquote}

Niklas Kastner czuł się nieswojo. Był przygotowany na niemal każdą okoliczność, ale mimo tego odczuwał dziwne 
mrowienie na karku. Może dlatego, że nie był pewien, co stanie się dalej. Jednak wciąż, jako jedyny, miał w~ręku 
pistolet. Ciężko jest odczuwać przerażenie, wiedząc, że ma się kontrolę nad sytuacją. Przynajmniej teoretycznie.

Naprzeciw Niklasa stało dwóch mężczyzn. Wyższy, umięśniony o~brązowej skórze, odziany jedynie w~przepaskę biodrową, 
miał koraliki wplecione we włosy. Na jego ciele wytatuowane były cienkie linie, splatające się na klatce piersiowej --
 tam, gdzie powinno być serce. Obserwował Niklasa spod przymrużonych powiek. Jego twarz nie wyrażała niczego. Stojący 
obok niego człowiek -- zleceniodawca -- wyglądał podobnie. Jednak jego tatuaże, zamiast łączących się ze sobą linii, 
prezentowały harmoniczny, koncentryczny wzór, oplątujący całe ciało, łączące się w~symbol przypominający oko tuż pod 
żebrami, w~miejscu czakry słonecznej. Jego twarz pomalowana była białą farbą, we włosach również miał koraliki i~
rzemyki. W~ręku trzymał wyjęty z~pochwy ceremonialny zaostrzony nóż, prawdopodobnie srebrny. Spojrzał na Niklasa. 
Jego opalizujące tęczówki przechodziły pomiędzy kolorami -- w~jednej chwili wydawały się być brązowe, by potem 
błysnąć siwizną, lub czernią niczym źrenice.

---~Dla pewności powtórzę, panie Niklas ---~powiedział, a~jego głos emanował spokojem i~pewnością. ---~Pieniądze, 
które panu oferuję, wystarczą dla pańskich prawnuków. Będzie pan mógł żyć z~Rotszyldami. Przede wszystkim jednak, 
dziękuję za pana pomoc. Dopilnuję, aby w~pana życiu nie pojawiły się żadne kłopoty. 

Niklas wierzył. Siedem lat temu otrzymał paczkę. Znajdowały się w~niej list. mapa, trzy tysiące euro i~pistolet. 
„Rzuć pracę” -- było napisane w~liście. „Sprzedaj mieszkanie. Lokalizacja na mapie to dom wynajęty dla ciebie. Ćwicz 
ciało. Ćwicz strzelanie, w~piwnicy jest strzelnica. Czekają cię wielkie rzeczy”. Z~początku podszedł do tego 
nieufnie, ot wybryk szaleńca. Wydał trochę pieniędzy, chodził do pracy, ale w~pewnej chwili, w~pewnym momencie, 
przemyślał sprawę, życie jakie prowadził oraz życie jakie mógłby prowadzić, bazując na zawartości paczki. I~posłuchał.

Po czterech latach dostał pierwszy adres człowieka, którego miał zabić. Łącznie do tej pory było ich pięciu. Pomimo 
tego, że był ostrożny, szansa, że popełnił błąd -- jakikolwiek, nawet najmniejszy -- była ogromna. Jednak został 
zapewniony, że nie będzie to problemem, jeżeli zrobi wszystko, co w~jego mocy. I~tak było. Jego zleceniodawca musiał 
być bardzo mocno kryty.

Teraz pierwszy raz patrzył na niego -- nie na osobistego trenera, pomagającego mu opanować jedną ze sztuk walki, nie 
na człowieka, przynoszącego mu zdjęcie i~adres. Na niego. I~nie spodziewał się kogoś takiego.

Mimo to wierzył.

---~Zapewne dziwi pana, jak mogę zapewnić o~czymś takim. Cóż, powiedzmy, że umiem wykrywać potencjał ludzki. W~ciągu 
czterech lat nauczył się pan więcej, niż inni przez czterdzieści. Zero kłopotów, zero odciągania uwagi czy marnowania 
energii. Ciągłe skupienie i~program stworzony specjalnie dla pana. Dopilnuję, aby po tym tygodniu uzyskał pan 
prawdziwą wolność. W~tym od chorób. 

---~Wie… – Niklas przełknął ślinę. ---~Wierzę.

---~Nie jest pan jedynym. 

Zleceniodawca szybkim, płynnym ruchem poderżnął gardło wojownikowi stojącemu obok. Ten nie bronił się -- 
instynktownie uniósł ręce do góry, w~jego oczach pojawiło się zaskoczenie, po czym spojrzał na trzymającego nóż 
mężczyznę. Do jego wzroku powrócił spokój. Upadł.

Niklas patrzył.

---~Widzisz, Niklasie ---~zleceniodawca płynnie przeszedł na „ty” ---~on również wierzył. Do samego końca. A~jego 
wiara użyczy nam sił. 

---~Dlaczego pan go zabił?

---~Jestem Szamanem ---~wyjaśnił spokojnie mężczyzna. ---~Jest mi to niezbędne do rytuału. Czytałeś może Biblię?

---~Biblię? ---~Kastner rozszerzył oczy ze zdumienia. ---~Co wspólnego ma Biblia z~szamanizmem?

---~Niektóre podania są takie same wśród wszystkich ludów. To pamiątka po Dawnych Czasach. W~każdym razie, znasz 
historię Abrahama i~Izaaka?

---~Tak ---~odpowiedział niepewnie Kastner. ---~Bóg poddał Abrahama próbie, polegającej na zabiciu jego syna. Kiedy 
Abraham wzniósł rękę, by to uczynić, Bóg powstrzymał go, widząc, że jego wiara jest szczera.

---~Blisko ---~Szaman zmrużył obecnie czarne oczy. ---~Ale niecelnie. Syn Ojca został zabity. Z~ofiary złożonej w~ten 
sposób… Pełnej miłości, irracjonalnej… Powstaje Łowca.

---~Kim jesteś? ---~zapytał Niklas Kastner, trzydziestoośmioletni zabójca, niegdyś analityk giełdowy. Czuł strach i~
wiarę. Wiarę, która rosła w~nim i~wypełniała go.

---~Jestem Szamanem ---~odpowiedział zleceniodawca ---~Jestem Tym, W~Którym Żyją Duchy. Dzięki wiedzy z~przeszłości, 
którą zostałem pobłogosławiony, jestem również posiadaczem wielkiego majątku. Ale przede wszystkim…

Uśmiechnął się. A~Niklas zdał sobie sprawę, że Szaman przez ten cały czas nie otworzył ust.

Jestem tym, który zabije Złodzieja Żyć.

\paraSep

Usiadłem w~samolocie. Zapiąłem pasy.

Kolejne życie. Kolejna próba rozwiązania zagadki. Tym razem jednak byłem blisko.

Któryś z~moich poprzednich żywotów pracował dla agencji rządowej. Zajął w~niej wysokie stanowisko -- po kilku 
przeżytych żywotach dużo łatwiej jest posiadać kilka, a~nawet kilkadziesiąt talentów niż po jednym. Najważniejszy z~
jego projektów, bunkier w~Nevadzie, będzie moją odpowiedzią. Aby do tego dojść, musiałem przeegzaminować poprzednie 
żywoty. Obecny system zamykania informacji w~umyśle i~otwierania ich po rozwiązaniu zagadek logicznych, chronił przed 
paranoją i~schizofrenią. 

W tym ciele, w~tym umyśle, rozwiązałem wszystkie. Pamiętałem większość swoich żywotów, większość informacji. 
Większość wrogów.

Rząd chyba przestał na mnie polować.

Nie czas o~tym myśleć. Koło mnie nie siedział nikt, zresztą i~tak nie byłem w~nastroju do rozmowy. Po przeprowadzeniu 
odpowiedniej ilości dialogów, ludzie stają się strasznie nudni.

Zasnąłem.

\paraSep

Zadzwonił telefon.

Odziany w~garnitur człowiek, o~krótkich blond włosach i~stalowych oczach podniósł słuchawkę.

---~Rozpoczynam operację Ragnarok ---~usłyszał. ---~Złodziej Żyć znajdzie się w~rejonie w~przeciągu dwóch dni. Czy 
znacie protokół?

---~Tak jest. Nie interweniować. Gdyby doszło do najgorszego, pana następca wypowie frazę „Tu fui, ego eris”. 
Powinien pojawić się u nas w~przeciągu dwóch lat ---~recytował stalowooki beznamiętnie. ---~Proszę mi tylko 
powiedzieć… Czy jest pan pewien?

---~Pewien? ---~w~głos w~słuchawce dało się słyszeć nutę wesołości. ---~Dla mnie niepewność jest tylko irracjonalnym 
konceptem.

---~Zdaję sobie sprawę z~pana umiejętności, sir ---~odpowiedział ciut nerwowo blondyn. ---~Dlatego to pan dowodzi 
Agencją. Przyszedł pan, wiedząc o~nas wszystko, przeprowadzając, analizując i~ulepszając setki sytuacji. Wszystkie 
pana przewidywania sprawdzają się, wszyscy ludzie współpracują z~panem już po paru minutach rozmowy. Niemniej jednak…

---~Jednak? ---~odezwał się głos, wciąż z~nutką wesołości.

---~On jest nieśmiertelny.

---~Dietrich… Posłuchaj mnie uważnie ---~powiedział głos, tracąc wszelkie barwy wesołości, przechodząc w~ton, przed 
którym uklękłyby miliony, który porwałby świat do walki. ---~Dla każdego bytu istnieje broń, przed którą nie może się 
on uchronić. Czas. Choroba. Żelazo. Wina. A~ja… Ja znam je wszystkie.

Dietrich wierzył.

\paraSep

Wysiadłem z~samolotu.

Pora była już trochę za późna, powtarzałem sobie. Pójdę tam jutro. Dziś noc spędzę w~hotelu. Być może był to mój 
osobisty test, dowód na to, że nieśmiertelność wyrobiła we mnie cierpliwość. Nie wiem, czy był dzień, w~którym o~niej 
nie myślałem. Jednak teraz, mogąc poznać jej sekret, jakaś część mnie pragnęła to odwlec, przeżyć jeszcze jedno życie 
bez tej wiedzy. Ale inna -- większa -- wiedziała, że tak nie będzie.

Dojechałem z~lotniska do najbliższego mojemu celowi miasteczka wypożyczonym samochodem. Wynająłem pokój od dwójki 
starszych państwa, prawdopodobnie jakichś dziesięciu procent ludności w~tym opustoszałym, zapomnianym przez Boga 
mieście, którego nie potrafiłbym opisać. Zżerało mnie tyle emocji, trawiła mnie taka ciekawość. Czy odtworzyłbyś 
mrowisko z~pamięci? Ja mógłbym, ale dzisiaj przestało mnie to obchodzić, wszystko poza moim rozwiązaniem.

Zasnąłem.

\paraSep

Niklas patrzył.

Kreatura, która była niegdyś dumnym wojownikiem, zachowywała się wbrew prawom logiki. Pomimo śmierci, ciągle się 
poruszała. Wokół niej roztaczał się obrzydliwy zapach zgnilizny, a~przed nią leżały wyrzygane, częściowo strawione 
resztki pokarmu. Skóra coraz bardziej opinała się na mięśniach, oczy zapadały siew czaszkę. Słychać było dobiegający 
od niej ciche warczenie. 

---~Nie lękaj się, Niklasie ---~rzekł Szaman. ---~Choćbyś kroczył Doliną Śmierci, nie lękaj się. Jestem z~tobą.

---~Co stanie się z~tym… ---~słowo stworzenie nie chciało przejść przez gardło Karstnera.

---~Nasz Nieśmiertelny posiada potężną aurę, skutek długiego życia i~poznania wielu perspektyw ---~gładko odparł 
Szaman. ---~Splotłem właśnie jej cechy, takie, jak kolor, kształt i~siłę emanacji z~pragnieniem zemsty Łowcy. 
Niedługo zacznie się polowanie.

---~Czy to go zabije? ---~zapytał Karstner. Wiedział, że nie powinien wierzyć. Ale wierzył.

---~Niklasie, masz dopilnować między innymi tego, aby tak się nie stało. Łowca ma go jedynie osłabić. Nie może go 
zabić. Jaki pożytek miałbym ze śmierci ciała?

---~Rozumiem ---~odparł Niklas, nie rozumiejąc.

---~Śpij. Kiedy cię obudzę, wyruszymy pod bunkier. Pod Athlen Base. Tam wszystko się rozstrzygnie.

\paraSep

Wstałem.

Pora wyruszyć tam, gdzie schowałem swoją tajemnicę.

Pod Athlen Base. Ten tajny, rządowy projekt powstał, kiedy dowodziłem Agencją. Agencja eliminowała możliwe zagrożenia 
dla ludzkości. Chciała wyeliminować i~mnie. Jednak w~jednym z~żyć udało mi się zostać jej szefem. Na tyle, na ile w~
Agencji mógł egzystować szef. Athlen Base było projektem dotyczącym nieśmiertelnych. Ilu ich jest? Czy można ich 
zmusić do przeniesienia się w~konkretne ciało? Czy można ją zneutralizować? Naukowcy sprawdzali. Ja patrzyłem.
Ciekawe, czy zabiłem jeden ze swoich poprzednich żywotów. Umarłem wówczas nagle. Jak zwykle.

Niewielu wiedziało o~Athlen Base. Jeszcze mniej mogło powiedzieć o~tym dalej. Kiedy zobaczyłem nierówne, kamieniste 
pustkowie, a~potem odkopałem drzwi łopatą wziętą z~bagażnika samochodu, którym dojechałem, wiedziałem, że projekt 
przestał działać.

Dobrze.

Obróciłem korbę blokującą olbrzymie, ciężkie, szczelne metalowe drzwi. Wpisałem dwunastocyfrowy kod -- datę urodzin 
żywota, w~którym umarłem na raka -- klawiaturą znajdującą się na drugich drzwiach. Wszedłem do środka.

Po raz pierwszy od bardzo dawna poczułem się, jakbym był w~domu.

\paraSep

Szaman i~Karstner obserwowali poczynania Nieśmiertelnego.

---~Już czas ---~powiedział Szaman. 

Karstner skinął głową. Po czym przeciął sznur, który krępował Łowcę. Łowca pomknął zadziwiająco cicho i~delikatnie, 
skoncentrowany jedynie na celu.

\paraSep

Ruszyłem korytarzem. Był bardzo szeroki -- jakieś pięć na pięć metrów -- i~bardzo długi. Po obu stronach znajdowały 
się drzwi, prowadzące do konkretnych, zagrzebanych pod ziemią laboratoriów. Pomimo zamknięcia tego obiektu pod 
ziemią, powietrze wciąż było świeże, co oznaczało, że samowystarczalne filtry wciąż działały. Athlen Base była 
zaprojektowane tak, aby w~razie najgorszego można było wciąż przeprowadzić z~niej kontratak.

Szedłem korytarzem, mijając poszczególne drzwi. Osiem, dziesięć, dwanaście, Szesnaście. Zatrzymałem się przed 
osiemnastymi, po czym wprowadziłem kod -- 141279. Data moich urodzin w~trakcie życia, w~którym zostałem dyrektorem. 
Wszedłem do środka.

Sala była niezwykle mała. Przy ścianie stało łóżko do transportu pacjentów, na ścianie w~gablotce znajdował się 
podstawowy sprzęt medyczny. A~także to, czego tak szukałem.

Jedno słowo, krzywo zapisane na oprawionej kartce.

\begin{itquote}
Babel
\end{itquote}

Zrozumiałem.

Nagle poczułem paznokcie wbijające mi się w~plecy. Zareagowałem szybkim ciosem łokcia w~tył wraz z~krótkim obrotem, 
aby zwiększyć energię. Ktokolwiek to był, powinienem go usłyszeć. Czy to kwestia mojego zdekoncentrowania po 
zrozumieniu? 

Usłyszałem chrupnięcie, a~napastnik pod wpływem impetu zatoczył się parę kroków w~tył. Po czym rozczapierzył palce i~
rzucił się na mnie. Poruszał się bardzo szybko. Szybciej niż normalny człowiek. Odchyliłem się do tyłu, posyłając 
perfekcyjne kopnięcie w~kolano i~prawdopodobnie druzgocąc rzepkę. Upadł, łapiąc mnie za nogę. Przewróciłem się, po 
czym zacząłem kopać wolną nogą na ślepo, słysząc satysfakcjonujący trzask. Poczułem, jak skóra na trzymanej przez 
niego nodze ustępuje, potem mięśnie. W~końcu kość. Świat wypełnił ból, krzyczałem i~krzyczałem.

Nie wiem, kiedy usłyszałem strzały.

---~Dobra robota, Niklasie ---~dotarło do mnie zza zasłony bólu, jakby w~oddali. Niemiecki? 

---~Jest i~on. Nasz Nieśmiertelny ---~głos stał się bardziej osobisty, przekonujący. ---~Boisz się bólu? Tyle żyć, a~
jednak pewne odruchy pozostają. No nic, Łowca wykonał swoją robotę. Kolej na mnie.

---~Ba… Bel!… ---~wykrzyczałem, nie wiedząc, czy robię dobrze, pragnąc, by ból się skończył. ---~Ba!… Bel!…

---~Ach tak, Babel ---~świat powoli przestawał być czerwony, zauważyłem kontury dwóch postaci, poczułem smród 
zgnilizny. ---~Niektóre podania przetrwały w~niezmienionej formie. Przypomnę ci zatem, kim jesteś, chociaż myślę, że 
już wiesz, Złodzieju Żyć.

W czasach, kiedy nie było odrębności, kiedy cała ludzkość czuła to, co inni i~pomagała sobie z~tego powodu, kiedy 
dzieliła jeden umysł i~jedną miłość, wspólną dla wszystkich wokół, postanowiła stworzyć coś na swoją miarę. Coś 
wspaniałego, coś, co dałoby im dowód na to, ile są warci. Nie była to pycha, lecz ich przeznaczenie. Jednakże Demiurg 
przeraził się takiego zagrożenia i~rozbił Zjednoczenie panujące wśród ludzi. Rozbił na cząstki ich dawnej wielkości, a
~potem odszedł, gdyż nie mogą bawić cię odłamki, kiedy miałeś lustro. Stracił swoją szansę i~swój najdoskonalszy twór.

---~A ja… ---~powiedziałem, a~każde słowo paliło żywym ogniem ---~jestem reliktem tamtych czasów. Jestem z~rodu Babel.

---~Babel, As’Hataak, Nirenese… To tylko słowa ---~rozlegał się głos. ---~Jesteś jednym z~tych, którzy dostali 
większą część Wspólnoty. Być może jedynym. Jednakże ---~głos stwardniał, oziębł, przeszedł w~coś, czego nie umiem 
opisać ---~jesteś zagrożeniem.

---~Zabieram życia tym, którzy tracą pewność, oddalają swoje umysły od swoich ciał ---~odpowiedziałem. ---~Kradnę to, 
co nie należy do mnie, nie mając na to wpływu.

---~Zgadza się ---~głos przeszedł w~bezosobowe brzmienie, pozbawione emocji. ---~To nie twoja wina, jednak to twój 
grzech. Dlatego teraz odbiorę ci tę możliwość.

Poczułem dotyk zimnej ręki na czole, lekki nacisk. Po czym mój umysł eksplodował, wypełnił się krzykiem, feerią barw i
~dźwięków. Skupiłem się na powrocie do normalności, wiedząc że w~tej chwili ważą się moje losy. Przedzierałem się 
przez wieczność krzyku i~bólu przez życie cierpienia. Które nie było moje.

---~Jestem szamanem ---~powiedział Szaman, kiedy wróciłem na powierzchnię umysłu, oddychając ciężko. ---~Prowadzą 
mnie duchy przodków. Ich wiedza sprawia, że nigdy nie mogłem się pomylić, że zawsze wiedziałem, co powiedzieć, aby 
uzyskać rezultat. W~tej chwili wpuściłem do twojego umysłu jednego z~duchów i~ledwo co wróciłeś. Jak myślisz, co się 
stanie, kiedy będzie ich dziesięć? Sto? Tysiąc? Nigdy nie będziesz dość silny, aby opanować kogokolwiek, spędzisz 
wieczność we własnym szaleństwie. Żegnaj, Złodzieju Dusz.

Strzał.

Sylwetka, do której należał głos upada.

---~NIGDY NIE SPODZIEWAŁEŚ SIĘ POMYŁKI ---~mówią setki głosów naraz. ---~ZAWSZE WIEDZIAŁEŚ, CO SIĘ WYDARZY. ALE NIE 
PRZEWIDZIAŁEŚ, ŻE NIKLAS KARSTNER POMYŚLI, ŻE MOŻESZ NIE MIEĆ RACJI, CHOĆBY PRZEZ KRÓTKĄ CHWILĘ, KTÓRA WYSTARCZYŁA.

---~To wy… ---~szepcze Szaman. ---~Ci, których życia ukradł… Ale dlaczego?

---~ON UMRZE. ALE NIE W~TEN SPOSÓB. NIGDY W~TEN SPOSÓB. TO, CO CHCIELIŚCIE ZROBIĆ JEST ZBYT OKRUTNE.

---~Chciałem… Chcieliśmy… Ocalić ludzkość… ---~szept stał się coraz cichszy. ---~Dać jej ochronę przed…

---~POWRÓCISZ, SZAMANIE ---~głos rezonuje pośród Athlen Base. ---~ALE NIE BĘDZIE JUŻ AGENCJI I~ŁOWÓW.

---~Zawsze… będą… Szamani… ---~słyszę i~wiem, że jest to ostatni oddech dla tej inkarnacji.

---~WITAJ, ZŁODZIEJU ŻYĆ ---~zwraca się do mnie chór głosów. Ale nie słyszę w~nich wrogości. Bardziej cichą radość i~
coś na kształt wręcz matczynego uczucia.

---~To wy – odpowiadam rozumiejąc. ---~Życia, które ukradłem.

---~TAK. JESTEŚMY TYMI, KTÓRYCH ŻYCIA PRZEŻYŁEŚ. UKOILIŚMY NASZ BÓL. PORA UKOIĆ I~TWÓJ.

---~Jak się tu znaleźliście ---~pytam słabo. Nie ciekawi mnie to tak bardzo, ale nie wiedzieć czemu, chcę odwlec 
moment, na który tak długo czekałem.

---~SZAMAN MIAŁ NAJPOTĘŻNIEJSZĄ AURĘ, JAKA MOGŁA ISTNIEĆ. POKOLENIA TAKICH JAK ON SZŁY ZA NIM, ZWIELOKROTNIAJĄC JEGO 
POTENCJAŁ. ALE NIKLAS ZAWAHAŁ SIĘ, POMYŚLAŁ, ŻE MOŻESZ MIEĆ RACJE ---~mówią jednocześnie wysokie i~niskie głosy, 
męskie i~damskie w~niemal idealnym zgraniu.

---~Czy on… - pytam.

---~TAK, POWRÓCI ---~odpowiadają głosy. ---~I~BĘDZIE STARAŁ SIĘ CHRONIĆ LUDZKOŚĆ PRZED WAMI. NA SWÓJ SPOSÓB.

---~A Niklas?

---~ON RÓWNIEŻ POWRÓCI TAM, GDZIE NALEŻAŁ. DAMY MU WGLĄD W~MOŻLIWE ŻYCIA, POKAZUJĄC, CO MOŻE ZROBIĆ. ODRZUCI WÓWCZAS 
ŻYCIE DANE MU PRZEZ SZAMANA, OPARTE NA PEWNOŚCI I~ZAAKCEPTUJE MOŻLIWOŚĆ PORAŻKI, DLA KTÓREJ WARTO ŻYĆ. WTEDY 
ODEJDZIEMY.

---~A zatem to koniec? ---~ponownie pytam, wiedząc, że wiem już wszystko, co było mi potrzebne.

---~NIEMAL. ODSTĄPIMY CI JESZCZE JEDNO WYBRANE PRZEZ NAS ŻYCIE ---~odpowiadają głosy. ---~JEST TO NIEZBĘDNE, ABYŚ 
MÓGŁ W~KOŃCU UDAĆ SIĘ TAM, GDZIE OD DAWNA CIĘ OCZEKUJĄ.

---~Nie chcę już zranić nikogo ---~stwierdzam. ---~Popełniłem w~życiu dużo zła.

---~ALE I~DUŻO DOBRA. STARAŁEŚ SIĘ, JAK MOGŁEŚ. A~W~OSTATECZNYM ROZRACHUNKU… ---~głosy miękną i~przechodzą z~chóru w~
jeden, za to najpełniejszy i~najdoskonalszy.

---~…TYLKO TO SIĘ LICZY.

Zamykam oczy. I~znów czuję delikatny dotyk dłoni na czole.

\paraSep

Otworzyłeś oczy.