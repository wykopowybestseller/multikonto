\chapter{Prolog}
\chapterAuthor{klik34}

\begin{itquote}
Spadam, znów lecę w~przepaść. Ciekaw jestem, ile jeszcze razy czeka mnie to samo. Nie potrafię zliczyć, jak wiele 
godzin spędziłem nad rozmyślaniem dlaczego ja, czemu to mnie wybrano. Za każdym razem czuję chłód, przeszywający ból i
~pustkę, po czym cierpienie rozpoczyna się na nowo. Komu zawiniłem? Jaki jest tego cel? Nie mam pojęcia, wiem tylko, 
że niedługo ponownie się obudzę, ponownie będę musiał stawić czoła codzienności, mając w~sobie całą gorycz kilkuset 
lat egzystencji. Żegnałem już wielu przyjaciół i~ukochanych. Najzabawniejsze jest to, że niektórzy z~nich żegnali 
mnie kilkukrotnie, niczego nie podejrzewając. Za każdym razem, kiedy miałem tylko możliwość, próbowałem się do nich 
zbliżyć, lecz nie dając nic po sobie poznać. Po pewnym czasie całkowicie przestało mi zależeć, stałem się egoistą 
myślącym tylko o~sobie. Był jednak pewien okres, gdy starałem się swoje przekleństwo wykorzystać dla dobrych celów: 
tu dla kogoś spełniłem największe marzenia, tam okradłem bank i~rozdałem gotówkę potrzebującym. Boże, jaki ja byłem 
głupi! Zapomniałem całkowicie, że każdy czyn ma swoje konsekwencje. Zupełnie jakbym nie oglądał Efektu Motyla, a~
robiłem to przecież tysiące razy. Ten, z~pozoru zwykły film analizowałem klatka po klatce i~to jakoś pozwoliło mi 
utrzymać jasność umysłu. Jasność umysłu… przez ten cholerny dar spędziłem trzydzieści lat w~zakładzie zamkniętym. 
Nikt mi nie uwierzył, bo niby kto by uwierzył w~to, że za każdym razem rodzę się na nowo? A~ty byś to zrobił?
\end{itquote}
