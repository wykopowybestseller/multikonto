\chapter{Życie VI}
\chapterAuthor{nt22}

---~Adam, czemu tak przeraźliwie krzyczysz, czy coś się stało?

---~Nic się nie stało, ale miałem straszny sen.

---~Co ci się śniło?

---~Czy ty musisz wszystko wiedzieć? Nie chcę o~tym mówić.

---~Jeśli nie chcesz o~tym rozmawiać, to nie, ale ostatnio coraz częściej miewasz złe sny.

---~Wiem, ale to jest moja sprawa i~nic ci do tego. Zresztą nie mam czasu, śpieszę się na pociąg.

---~Dobrze, dobrze… Znowu się denerwujesz, a~ja chcę ci pomóc.
 
Taaaaaaa, pomóc… Dobre sobie – pomyślałem.

---~Haniu! wezmę szybki prysznic, a~ty przygotuj mi proszę, śniadanie.

Od kiedy mam koszmary nocne, budzę się spocony jak mysz. Nie wiem co się dzieje, ale to nie jest normalne, z~dnia na 
dzień czuje się coraz gorzej. Czasami wydaje mi się, że już dłużej tego nie wytrzymam i~coś sobie zrobię. Dziwię się 
mojej żonie, że jest taka spokojna, w~końcu zabójstwo człowieka to poważna sprawa, a~ona zachowuje się jakby nic się 
nie stało. Cholera, w~końcu to ona zabiła, a~nie ja!
 
Ciężko się żyje z~morderczynią pod jednym dachem. Gdybym zeznał wtedy, jak było naprawdę, to dzisiaj siedziałaby w~
więzieniu, a~tak siedzi tam niewinna osoba. Co ja najlepszego narobiłem…
 
Dobra, kończę brać ten prysznic i~ubieram się, bo znowu się spóźnię do pracy.

---~Znowu z~salcesonem? Przecież wiesz, że nie lubię.

---~Jedz to co jest i~nie marudź! Dobrze wiesz, że od czasu jak straciłam pracę, to nie mamy pieniędzy. Zresztą od 
dłuższego czasu nic nam nie wychodzi, nie wiem co się z~nami dzieje.

---~Wiesz co? Już mi się odechciało. Wychodzę! Zepsułaś mi humor, znowu robisz ze mnie wariata!

---~Co ja takiego zrobiłam? Z~ciebie wariata?

---~Jak zwykle nic, mniejsza o~to!
 
Ta kobieta jest po prostu żałosna, zabiła moją koleżankę z~powodu chorobliwej zazdrości i~udaje, że nie wie co się 
stało, zachowuje się tak, jakby ta sytuacja nie miała miejsca. Gdybym miał siłę, to załatwiłbym sprawę po męsku, ale 
do tego trzeba być mężczyzną. Niestety ja nim nie jestem, każda kobieta z~którą się wiązałem owijała sobie mnie wokół 
palca.
 
Od wielu lat sytuacja wygląda identycznie, ona zawsze psuje mi humor, z~jej strony nigdy nie spotkało mnie nic 
dobrego, a~na dodatek nie potrafi mi nawet przygotować porządnego śniadania! Żeby chociaż była dobra w~łóżku… Co ja 
mówię, przecież ona jest zimna jak lód! Zresztą od wielu lat nie sypiamy ze sobą. Czasami odechciewa mi się żyć przez 
to wszystko. Dlaczego nie spotyka mnie nic dobrego? Dlaczego? Pewnie dlatego, że jestem głupi! Kiedy inni mnie 
ostrzegali przed nią, ja się uparłem i~ożeniłem się z~nią, wbrew wszystkim. Teraz mam za swoje… Cierpię za swoją 
głupotę i~nie liczę na szczęście. Jestem realistą.

---~Dobra, koniec użalania się nad sobą. Idę, bo się spóźnię na pociąg.
 
15 minut później siedziałem już w~pociągu. Jeżdżę nim codziennie od kilkunastu lat do pracy, której nienawidzę. 
Niestety nie mam wyjścia, pracować muszę, za coś trzeba żyć. Hania nie pracuje, więc jestem jedynym żywicielem rodziny.

---~Dzień dobry Adamie! 

---~Dzień dobry ---~odpowiadam, mimo tego, że nikogo nie widzę.

Nagle widzę twarz jakby znajomą, ale jeszcze nie wiem kto to jest. Skupiam myśli, kto to może być, no nie mam 
pojęcia. Mam 43 lata, a~pamięć jak dziewięćdziesięciolatek, co się ze mną dzieje? Jedno jest pewne, kobieta jest 
bardzo piękna, jak z~bajki. Miło popatrzeć na ładną kobietę.

Niestety, ja na co dzień mam wstrętną ropuchę, której nie cierpię. Mniejsza o~to…

---~Adam, to Ty?

---~Kim pani jest?

---~Jestem Julia, chodziliśmy razem do szkoły podstawowej.

---~Przepraszam, ale nie poznałem cię w~pierwszej chwili.
 
Tak szczerze, to w~ogóle jej nie poznałem, kiedyś nie była zbyt urodziwą kobietą, a~teraz wygląda olśniewająco, 
rewelacyjnie, bosko, cudownie, wspaniale, kobieco, po prostu super!

---~Muszę przyznać, że jesteś jeszcze piękniejsza niż kiedyś.

Skłamałem, bo kiedyś była raczej brzydka, ale nie muszę jej o~tym mówić.

---~Dziękuję, a~ty nic się nie zmieniłeś ---~nie wiem, czy to uznać za komplement, ale chyba nie, bo nie byłem nigdy 
zbyt urodziwym facetem, to wyjaśnia dlaczego mam nieznośną i~brzydką żonę.
 
PIK, PIK, PIK…
 
---~Przepraszam cię na chwilę, dostałem SMS-a. To nic takiego, telefon z~sieci ---~znowu skłamałem.

\begin{itquote}
Jak będziesz wracał z~pracy kup ziemniaki.
\end{itquote}

Ta kobieta na odległość wyczuwa, kiedy może spotkać mnie coś dobrego i~wysyłając taką wiadomość sprawia, że w~środku 
cały się trzęsę. To nie jest normalne.

---~Co tam u ciebie? ---~zapytałem.

---~Jestem rozwódką, kilka lat temu rozwiodłam się. Miałam męża tyrana, z~którym nie dało się wytrzymać.

---~Ja też jestem w~trakcie rozwodu, nie układa mi się z~żoną ---~skłamałem kolejny raz.

Zachowuję się tak już od wielu lat, żyję w~ciągłym kłamstwie, ale nie mam innego wyjścia, nie jestem już w~stanie żyć 
normalnie. Mam dwie możliwości na uzyskanie świętego spokoju: pierwsza to śmierć, ale jestem na to za słaby, nie mam 
siły żeby założyć sobie smycz na szyję i~tak po prostu odejść. Opcja druga, to rozejść się z~kobietą, której 
nienawidzę, ale na to także nie mam sił. Najgorsze jest to, że tkwię w~tym całe lata i~nie robię nic żeby zmienić 
swój los.

---~Przepraszam cię, ale niedługo będę wysiadał.

---~Podasz mi swój numer telefonu?

---~Tak, 505-678-…

---~Dziękuję.

---~W~najbliższym czasie się do ciebie odezwę.

---~OK, będę czekała na kontakt.

Dzień w~pracy, której tak nie cierpię minął bardzo szybko, może dlatego, że cały czas miałem w~głowie dzisiejszy 
poranek i~spotkanie z~koleżanką z~podstawówki, która przez niemal trzy dekady z~brzydkiego kaczątka, stała się 
pięknym łabędziem. Niestety teraz wracam do domu, do kobiety, której nienawidzę, której się boję -- zresztą ilekroć 
przypomnę sobie o~niej, albo spojrzę na nią, kiedy jest w~zasięgu mojego wzroku, czuję strach, niechęć, obrzydzenie.

Teraz idę na pociąg, potem kilka minut pieszo i~jestem już w~domu. Z~oddali widzę szary blok z~wielkiej płyty -- 
prawdę mówiąc od kiedy pamiętam, ten widok mnie przytłacza. Wiem, że po przekroczeniu drzwi mojego mieszkania 
atmosfera się zagęści, będzie niemiło jak zawsze.

---~Cześć Adam, jak było w~pracy?

---~Normalnie, a~jak miało być?

---~Kupiłeś ziemniaki?

---~Nie, na śmierć zapomniałem, kupię jutro.

Cały dzień miała na to, żeby to zrobić, ale nie zrobiła, jak zawsze wszystko jest na mojej głowie. Ona zrobi 
wszystko, żeby wbić mi szpilkę, zrobi wszystko, żebym poczuł się gorzej, może robi to podświadomie, ale faktem jest 
to, że czuje się w~tym małżeństwie po prostu źle.

Wieczór minął jak zawsze na niczym. Próbowałem wysłać SMS-a do Julii, ale od kilku godzin nie udało mi się to z~tego 
względu, że od zawsze po przyjściu z~pracy kładę moją komórkę na segmencie, jeśli jej tam nie będzie chociaż przez 
chwilę, to moja żona niczym hitlerowiec urządzi mi piekło, odechce mi się wszystkiego, kontaktu z~kimkolwiek, dlatego 
wolę uważać. Wiadomość do koleżanki z~podstawówki wyślę zaraz po tym, jak Hania położy się spać – na szczęście chodzi 
spać szybko, pewnie jest zmęczona nieróbstwem i~knuciem. Parszywa gnida.

SMS-a wysłałem zaraz po tym, jak Hania położyła się spać, dostałem szybko odpowiedź. Umówiliśmy się na jutro -- wezmę 
wolne na żądanie i~tyle. Nie mam nic do stracenia. Już długo nie czułem się tak, jak się teraz czuję. Ciężko opisać 
to uczucie. Jest ono raczej przyjemne.

Położyłem się spać trochę po północy, byłem bardzo zmęczony. To był dzień inny niż wszystkie.

\paraSep

Następnego ranka obudziłem się bardzo szybko, dwie godziny wcześniej niż normalnie -- tej nocy na szczęście nic mi 
się nie śniło i~nie wydzierałem się jak stare prześcieradło. Bałem się, że ona się czegoś domyśli, bałem się wstać z~
łóżka, wiem o~tym, że Hania by usłyszała, ona nie śpi jak człowiek, śpi na tak zwanej czujce i~dzięki temu kontroluje 
wszystko to co robię, wie o~mnie wszystko. Dlatego wolę sobie poleżeć dwie godziny, wstanę o~tej porze co zawsze, mam 
tylko nadzieję, że nie zrobię czegoś głupiego z~samego rana, bo mogę zaliczyć wpadkę i~jeszcze mnie kropnie ta 
wariatka, a~tego przecież nie chcę.

---~Haniu idę pod prysznic, zrób mi proszę coś do zjedzenia, śpieszę się do pracy.

---~Dobrze, kochanie.

Kochanie? Ona do mnie nigdy nie mówi kochanie, mam nadzieję, że niczego nie podejrzewa i~się tylko przejęzyczyła. 
Telefon leży na segmencie jak zawsze, ja pod prysznicem jak co dzień, wszystko jest OK. Przynajmniej mam taką 
nadzieję. Ona ma wiele zmysłów i~jest nieobliczalna.

Koniec ociągania się, ubieram się, zjem śniadanie i~lecę na pociąg. Muszę pamiętać, żeby po drodze zadzwonić do 
szefa, wezmę dzień wolny na żądanie, raz na kilkanaście lat można.

---~Co dzisiaj jest na śniadanie?

---~Bułka w~mleku.

Znamy się tyle lat, przecież wie, że nie cierpię bułki w~mleku! Czy ona robi to specjalnie? Pewnie tak, ale nie mogę 
dać po sobie poznać, że mi nie smakuje. Nie dzisiaj, nie w~taki dzień. Będę udawał, że bułka w~mleku jest 
perfekcyjna, wręcz idealna, boska.

---~Śniadanie było wyjątkowo smaczne, uciekam do pracy.

---~Uciekaj, uciekaj, do zobaczenia, do wieczora.

Lecę na pociąg, bo się spóźnię, a~dzisiaj bym tego nie chciał. Do szefa zadzwonię z~pociągu, powiem, że z~powodów 
rodzinnych dzisiaj nie będzie mnie w~pracy. Nie potrafię kłamać, boję się, że przez to ktoś się czegoś domyśli, 
chociaż dzisiaj jestem zmotywowany, dzisiaj będę kłamał ile wlezie, byle tylko spotkać się z~Julią. W~tym momencie 
myślę sobie: będzie, kurwa, dobrze!

Udało się, szef dał mi dzień wolny, teraz jadę do kawiarni, która znajduje bardzo blisko mojej pracy, to właśnie tam 
umówiłem się z~Julią. Niestety nie mogłem jej zaprosić w~inne miejsce, bo nie mam zbyt wiele pieniędzy -- od lat 
Hania zabiera mi całą wypłatę.

Wysiadłem z~pociągu, droga do miejsca w~którym się umówiliśmy nie jest zbyt długa, ale mimo wszystko muszę się 
zastanowić co jej powiem, cholera powinienem kupić jej kwiaty, no ale nie bardzo mam za co, mam nadzieję, że zrozumie 
moją sytuację. Zresztą lepsza kawa i~ciasto w~kawiarni, niż bukiet kwiatów, które za chwilę wyrzuci do kosza na 
śmieci. Przynajmniej tak myślę, może się mylę, sam nie wiem, chyba już zapomniałem jak się postępuje z~kobietami. Już 
jestem na miejscu, nie ma jej jeszcze, mam nadzieję, że zaraz przyjdzie. Po chwili słyszę:

---~Cześć Adam.

---~Cześć Julia.

---~Jak pięknie dzisiaj wyglądasz, dobrze ci w~czerwonym.

---~Ty też wyglądasz całkiem nieźle.

Dobrze wiem, że skłamała, ja doskonale wiem jak wyglądam, według mnie wyglądem przypominam raczej menela, niż 
człowieka, który wygląda „całkiem nieźle”, plus dla niej za to, że potrafi być miła.

---~Wejdziemy do środka? ---~zapytałem.

---~Pewnie, że tak.

Lokal może nie należy do najpiękniejszych, ale dla mnie najważniejsze jest to, że jest tutaj Julia, która wygląda 
wręcz olśniewająco. Swoją drogą, niektóre kobiety są jak wino.

---~Czego się napijesz? ---~zapytałem Julię.

---~Kawę poproszę ---~odpowiedziała szybko.

---~Zaraz wracam, idę złożyć zamówienie.

Poprosiłem o~dwie kawy i~dwa małe ciastka -- zresztą na więcej mnie nie stać. Mam nadzieję, że mi wystarczy 
pieniędzy, jeśli nie, to pożyczę od Julii, ale to się nie może zdarzyć, jakby to wyglądało, 43-letni facet zaprasza 
kobietę do kawiarni i~pożycza od niej pieniądze, aż strach pomyśleć.

---~Proszę bardzo Julio.

---~Dziękuję.

---~Nie ma za co.

---~Co tam u ciebie słychać?

Spotkanie mijało szybko i~było bardzo miło -- rozmawiało się nam super. Mimo tego, że ja po latach bycia z~tym 
potworem, odzwyczaiłem się od normalnych rozmów z~kobietami, to śmieszne, ale odzwyczaiłem się od obcowania z~nimi. 
Zapomniałem jak to jest być z~kobietą.

---~Julio, będę musiał już powoli uciekać.

---~Tak szybko, czy coś się stało?

---~Nic się nie stało, wręcz przeciwnie.

---~To czemu chcesz już uciekać?

---~Niestety muszę być w~domu o~normalnej porze, Moja żona to wariatka i~mimo tego, że się rozwodzimy robi mi jazdy. 
Nigdy nie miałem z~nią łatwo, dlatego się rozwodzę.

---~Dobrze, rozumiem.

Musiałem skłamać Julii, mimo tego, że bardzo tego nie chciałem, ona nie mogła się dowiedzieć jak jest, zależy mi na 
tej znajomości, w~końcu mam możliwość obcowania z~normalną kobietą.

---~Do widzenia, Julio.

---~Do widzenia, Adam. Odezwiesz się jeszcze?

---~Pewnie, że tak, możesz być tego pewna, a~ty się odezwiesz?

---~Tak, odezwę się na pewno.

---~To świetnie, biegnę na pociąg, żegnaj.

---~Żegnaj.

Podróż do domu minęła jak zawsze, jak co dzień, nie zdarzyło się nic nadzwyczajnego. Zastanawiałem się tylko jak będę 
się zachowywał w~domu, czy Hania niczego się nie domyśli, nie powinna, nic przecież się takiego nie stało. No chyba, 
że ja zrobię coś głupiego.

---~Hej Haniu.

---~No cześć.

---~Jak w~pracy? Wszystko dobrze?

---~Tak, wszystko dobrze.

Co ona się tak wypytuje, nagle się ciekawska zrobiła, ale na szczęście nie powiedziałem nic głupiego, nie zająknąłem 
się, jest dobrze, przynajmniej mam taką nadzieję, bo po niej nigdy nic nie wiadomo. Ciężko przewidzieć jak się 
zachowa, ale żyję w~tej niepewności już kilkanaście lat, także zdążyłem się przyzwyczaić i~nauczyłem się z~tym żyć, w~
sumie nie miałem wyjścia.

Wieczór minął jak zawsze na niczym, dużo myślałem o~Julii, wysłałem do niej nawet SMS-a w~czasie gdy Hania brała 
prysznic. Cały czas myślę o~tej sytuacji i~tak szczerze mówiąc nie wiem co począć, czy spotykać się dalej z~Julią, 
czy ze strachu przed żoną dać sobie spokój. Jedno jest pewne, Julia to przesympatyczna kobieta, z~którą można miło 
porozmawiać na każdy temat.

---~Haniu, idę pod prysznic, a~ty przygotuj mi proszę kanapki.

---~Dobrze, zaraz ci przygotuję coś do zjedzenia

---~Dziękuję.

Dobra, wezmę szybki prysznic, coś zjem i~kładę się spać. Niby nic takiego się dzisiaj nie stało, a~mimo to jestem tak 
zmęczony, jakbym cały dzień w~kamieniołomach przepracował!

---~Adam, siadaj do stołu.

---~Już idę, co jest dzisiaj na kolację?

---~Chleb ze smalcem i~solą, niestety nic innego nie ma w~domu.

---~Czy Ty jesteś normalna? Czy ja jestem zwierzęciem? Chcę jeść normalnie, jak człowiek.

---~Żeby jeść jak człowiek trzeba na to zasłużyć!

---~Znowu Ci odbija, przecież ja nic takiego nie zrobiłem!

---~Nic nie zrobiłeś? W~takim razie powiedz mi, kim jest Julia?

---~To tylko koleżanka.

---~Przeczytałam SMS-y i~z nich wynika, że to jest ktoś więcej niż koleżanka!

---~Znowu grzebałaś mi w~telefonie? Ty kurwo, ja ci pokażę, popamiętasz mnie!
 
ŁUBUDU, ŁUBUDU, ŁUBUDU…

Ma za swoje, mam nadzieję, że się jej nic nie stało, od takiego uderzenia mogła dostać wstrząśnienia mózgu, ale 
znając jej szczęście pewnie wyjdzie z~tego bez szwanku.

---~Hania wstawaj, Hania wstawaj, słyszysz mnie? Hania!

Muszę sprawdzić jej tętno. Czemu ona się nie odzywa, pewnie udaje, żeby mnie nastraszyć. Ona zrobi wszystko, żebym 
się zdenerwował, robi mi to ciągle od lat. Moje cierpienie sprawia jej przyjemność. Najgorsze jest to, że nie czuję 
tętna!

---~Hania wstawaj! Nie udawaj!

O kurwa. Wygląda na to, że to ścierwo nie żyje. Zgniję w~pierdlu do końca życia!

Już wiem co zrobię, poćwiartuje tę szmatę, a~w~nocy wyniosę ją do lasu, mam nadzieję, że zjedzą ją zwierzęta, to jest 
chyba jedyne sensowne wyjście z~tej sytuacji, innego pomysłu nie mam.

DRYŃ, DRYŃ, DRYŃ

O kurwa policja, już tutaj są, a~nie to dzwonek mojego telefonu!

---~Adam, to ty? Z~tej strony Julia.

---~Tak, to ja, czy coś się stało, że dzwonisz o~tej porze?

---~Tak.

---~Możesz mi powiedzieć co takiego?

---~Kocham Cię!

---~Ja też cię kocham, szczerze mówiąc bałem się ci to wyznać. Nie bardzo mogę teraz rozmawiać, ale mogę odezwać się 
do ciebie jutro?

---~Tak, możesz. Rozumiem, że twoja żona jest w~domu?

---~Nie, nie, żony nie ma, wyjechała, mam tutaj małą awarię i~chcę to naprawić.

---~Nie ma problemu, czyli mam czekać na telefon?

---~Tak, odezwę się na sto procent.

---~Kocham cię.

---~Też cię kocham.

---~Dobranoc.

---~Dobranoc.

\paraSep
 
JAPONIA
 
---~Hanako wstawaj, jest już 10.00.

---~Wiem, wiem, jest już późno.

---~Byłaś bardzo nerwowa tej nocy, strasznie krzyczałaś…

---~Przepraszam, ale miałam koszmar.

---~Właśnie przeczytałam pierwszy rozdział książki, którą ostatnio zaczęłaś czytać i~muszę ci powiedzieć, że to chyba 
nie jest książka dla osób w~twoim wieku.
